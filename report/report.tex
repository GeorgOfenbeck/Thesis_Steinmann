\documentclass[a4paper,12pt]{article}
\usepackage[top=2cm]{geometry}
\usepackage{color}
\usepackage{listings}
\usepackage{calc}
\usepackage{booktabs}
\usepackage{subfig}
\usepackage{amsmath}
\usepackage{dblfnote}
\usepackage{pfnote}
\usepackage{makeidx}

\usepackage[colorlinks,bookmarks,pagebackref]{hyperref}
\usepackage{graphicx}
\newlength{\imgwidth}
\newlength{\imgleftoverhang}
\newcommand{\umlDiagram}[1]{%
	\settowidth{\imgwidth}{\includegraphics{out/diagrams/#1.pdf}}%
	\setlength{\imgwidth}{\minof{0.5\imgwidth}{\textwidth}}%
	\par\vskip0.5cm\noindent\makebox[\textwidth][c]{%
	\includegraphics[width=\imgwidth]{out/diagrams/#1.pdf}%
}\vskip0.5cm}

\newcommand{\umlFloat}[2]{
\begin{figure}[htbp]
\umlDiagram{#1}
\caption{#2}
\label{#1}
\end{figure}
}

\newcommand{\graphNoVskip}[1]
{
	\settowidth{\imgwidth}{\includegraphics{graphs/#1.pdf}}
	\setlength{\imgleftoverhang}{0.1\textwidth}
	\setlength{\imgwidth}{1.205\textwidth}
	\hfuzz50pt
	\noindent\hskip-\imgleftoverhang
	\hskip-0.25in\includegraphics[width=\imgwidth]{graphs/#1.pdf}
	\hfuzz0.1pt
}
	
\newcommand{\graph}[1]
{
	\par\vskip0.5cm
	\graphNoVskip{#1}
	\vskip0.5cm
}

\newcommand{\graphFloat}[2]
{
\begin{figure}[htbp]
\graphNoVskip{#1}
\vskip-0.5cm
\caption{#2}
\label{#1}
\end{figure}
}

\newcommand{\graphFloatTwo}[5]
{
\begin{figure}[htbp]
\subfloat[#3]{\graphNoVskip{#2}}\\
\subfloat[#5]{\graphNoVskip{#4}}
\caption{#1}
\label{#2}
\end{figure}
}

\newcommand{\umlFloatCap}[3]{
\begin{figure}[tbh]
\umlDiagram{#1}
\caption[#2]{#3}
\label{#1}
\end{figure}
}
\newcommand{\umlRef}[1]{\autoref{#1}}

\newcommand{\class}[1]{\textbf{#1}}
\newcommand{\method}[1]{\textsf{#1}}

\newcommand{\err}[0]{\ensuremath{\text{err}}}

\lstset{
  showstringspaces=false,
  breaklines=true,
  tabsize=4,
  numbers=left, stepnumber=2, numberstyle=\tiny, numbersep=10pt
}

\lstdefinestyle{pseudoCode}
{%%stringstyle=\textit
}

\sloppy
\renewcommand{\floatpagefraction}{0.7}
\renewcommand{\dblfloatpagefraction}{0.7}
    
\title{Applying the Roofline Model}
\author{Ruedi Steinmann}
\date{\today}

\begin{document}
\maketitle

\begin{abstract}
To increase CPU performance despite limited clock speeds (due to several
phisical issues), CPUs become more and more complex. This raises the difficulty
to understand their behavior, which is important for software performance
optimization.

In this context, insightful yet easy to understand performance models are of
great value. The Roofline Model premises to fulfill these criteria.

Unfortunately, to our best knowledge there is no tool available to measure the
required quantities and generate Roofline Plots. We built such a tool, evaluated
the quality of it's measurement results and generated Roofline Plots of various
code samples.

The tool is freely available and intended to be used by a wider audience.
\end{abstract}

\tableofcontents

\section{Introduction}
'The free lunch is over' \cite{FreeLunchIsOver} In the past decades, the CPU
manufactures were reliably able to increase the clock speed. However, this has
become harder and harder due to not just one but several physical issues. To
increase performance despite the limited clock speed, the CPUs become more
and more complex.

From the perspective of software performance optimization, the increased
complexity made performance observations harder to understand. Since the trend
to more complex CPUs is likely to continue in the future, an insightful and
easy-to-understand performance model would be of high value.
 
The roofline model \cite{Roofline} promises to fulfill these criteria. The key
observation which led to this model is that, for the foreseeable future, the
transfer bandwidth between CPU and memory will often be the limiting factor of
system performance. This is called the memory bottleneck.

In order to use the arithmetic units found in the CPU close to their peak
performance, some of the data loaded from memory has to be used in multiple
operations. This is enabled by the caches found in all modern computer
systems. A cache is a relatively small memory area which has a low access
latency and a high bandwidth. All data loaded from memory is saved in the cache,
evicting the least recently used data present in the cache. 

If the implementation of an algorithm manages to largely operate on data which
can be held in the CPU caches, off-chip bandwidth will not be an issue. 
Unfortunately it is often difficult to organize the data access
pattern of an algorithm in such a way that repeatedly accessed data is not
wiped out of the cache by intermediate data accesses.

During the optimization of an algorithm, many changes have an effect on both
performance and data access pattern. It is therefore beneficial for a
performance model to capture both aspects.

Towards this goal, the roofline model introduces the term 'operational
intensity' defined as the number of performed operations per byte transferred
between CPU and memory. The exact definition of operation in the context of
this definition is problem specific, but floating point operation is often used.

%\graphFloat{yonah/overview}{Overview of different Kernels}
% (fig.\ref{yonah/overview})

The roofline model combines the operational intensity on the $x$ axis with the
performance on the $y$ axis in a 2D graph . Each
measurement results in a point on the plot. If a measurement is repeated
for different problem sizes, the resulting points in the roofline plot
can be connected by a line to a series.

The peak floating point performance appears as a horizontal line. There can be
different bounds depending on the technology or the number of cores used. 

The memory bandwith can be shown in the roofline plot, too. If the
implementation of an algorithm transfers $v$ bytes (transfer volume) and
performs $\Omega$ operations within $t$ cycles (the unit of time does not
matter, could be seconds as well), it will be placed at $x=\Omega/v$
(operational intensity) and $y=\Omega/t$ (performance). The data throughput
between CPU and memory is $\tau=v/t$, which {\bf can be derived from performance
and operational intensity}:

$$
\tau=\frac{v}{t}
=\frac{\Omega v}{\Omega t}
=\frac{\Omega}{t}\frac{v}{\Omega}
=\frac{\Omega}{t}(\frac{\Omega}{v})^{-1}
=yx^{-1}
=\frac{y}{x}
$$

The throughput is bound by the bandwidth ($\tau<\beta$), which leads to the
following limit for performance and operational intensity:

\begin{align*}
\tau=\frac{y}{x}&<\beta\\
y&<x\beta 
\end{align*}

Roofline plots use a log-log scale. This enables the presentation of data in a
huge range. In addition, the bandwidth line appears as a line of unit slope,
offset towards the top left of the graph proportional to the logarithm of the
memory bandwidth.
\footnote{ In order to grasp the appearance of the bandwidth line, we have to
map the line to a cartesian coordinate system .
The coordinates $x'$ and $y'$ in the cartesian coordinate system are in the
following relationship to the log-log coordinates:

\begin{align*}
x'=\ln(x)&;\; e^{x'}=x\\
y'=\ln(y)&;\; e^{y'}=y\\
\end{align*}

The line becomes:

\begin{align*}
y&=\beta x\\
e^{y'}&=\beta e^{x'}\\
y'&=\ln(\beta e^{x'})\\
&=\ln(\beta) + \ln(e^{x'})\\
y'&=\ln(\beta) + x'\\
\end{align*}
}

As for the peak performance, there can be multiple bounds, depending for example
on the number of cores used, the memory type or the access pattern (random vs.
linear).

The horizontal and diagonal lines, called performance and bandwidth ceilings,
give the roofline model it's name. The point where the lines intersect is called
the ridge point.

The power of the model comes from the ability to show performance and
effects of the data access pattern at the same time. To illustrate this point,
we will try to solve the following problem: Given a matrix $M$ of size
$n\times m$ ($m$ rows, $n$ columns), we would like to perform the following
transformation: $M_{ij}=3j^2M_{ij}$.

The straight forward approach is to use the following loop: (Variant 0)
\begin{lstlisting}[language=C] 
for (int i = 0; i < m; i++) 
for (int j = 0; j < n; j++)
M[i*n + j] = 3*j*j*M[i*n + j];
\end{lstlisting}

In an attempt to optimize the algorithm, we could switch the order of the loops
to avoid repeated computations of the term $3j^2$: (Variant 1)

\begin{lstlisting}[language=C] 
for (int j = 0; j < n; j++) {
  int tmp = 3*j*j;
  for (int i = 0; i < m; i++) 
    M[i*n + j] = tmp*M[i*n + j];
}
\end{lstlisting}

We measured the two variants with $m=1e6$ and $n=4$ on an arbitrary computer and
got the following performances:
\begin{center}
\begin{tabular}{lr@{.}l}
\toprule
Variant & \multicolumn{2}{r}{Performance} \\
\midrule
0 & 0&193\\
1 & 0&026\\
\bottomrule
\end{tabular} 
\end{center}

Using just this information we have no clue why our modification degraded
performance. 

\graphFloat{yonah/intro}{Roofline Plot of the Introduction Example}
Using the roofline model we get figure \ref{yonah/intro}. We see that the
operational intensity decreased by an order of magnitude for variant 1.
Since both variants operate close to the peak memory bandwidth, it is clear that
the performance has to fall together with the operational intensity. In addition
the plot makes clear that the only way to get a substantial speed improvement is
to increase the operational intensity. Reviewing our modification, we can see
that we moved from a row major to a column major access pattern which is well
known to potentially reduce operational intensity.

Generally, if the roofline point of an algorithm lies to the left of the
ridge point, the only way to reach peak performance is to increase the
operational intensity. On the other hand, if a point lies to the right of the
ridge point,  memory bandwidth is not the limiting factor. If the kernel does not
reach the peak performance, there must be other performance limiting factors.

Apart from guiding the optimization process, the roofline model can be used to
analyze different computer systems. When comparing the roofline graphs of
different machines, the position of the ridge point is an important clue of how
difficult it is to reach a good performance. The further the ridge point lies to
the right, the higher is the operational intensity required to achieve peak
performance.

\subsection{Goals of the Thesis}
{\bf This thesis aims to create a tool capable of easily creating roofline
plots. In addition it should provide a measuring infrastructure which can be
used in a broader context.}

Code to be measured can come either in the form of a relatively isolated
routine, typically containing a single kernel, or in the form of a larger
program, containing multiple different kernels. While routines are treated as
atomic units, it is desirable to split the larger program into it's different
algorithms and do the analysis for each algorithm separately. In the rest of
this thesis, we will use the term kernel for a routine or part of a program
which is analyzed on it's own.

Many kernels are multi threaded in order to take advantage of CPUs with multiple
cores. This is achieved by splitting the algorithm into multiple threads which
can be executed in parallel. Our tool should be able to handle such kernels.

To get the data points for a kernel, we need to measure it's performance and
the operational intensity.

Performance is defined as amount of work per unit of time. The time required to
do the work has to be measured. The amount of work and how it is defined is
determined by the actual kernel. 

Operational intensity is defined as operations per byte of data traffic. Since
the operational intensity cannot be measured directly, we have to measure the
operation count and the data traffic.

The point where data traffic is measured is typically between the last level
cache of the processor and the DRAM, but it could be measured as well between
the different cache levels or between L1 cache and processor.

To obtain the peak performance and memory bandwidth lines for the roofline model
we need to create micro benchmarks. They are typically designed to transfer as
much data per unit of time for memory bandwidth lines and to execute as may
operations per unit of time for peak performance lines as possible. Thus the
measurement capabilities needed to evaluate the actual problems can be reused
for the benchmarks.

In summary, the following quantities need to be measured:
\begin{itemize}
\item Execution Time
\item Memory Traffic
\item Operation Count
\end{itemize}

The accuracy and precision of the measurement results will be evaluated and
possibly improved by statistical techniques.

The results will be used to automatically generate roofline plots. The
results of different kernels and sizes can be combined into a single plot.

The tool is written for the x86 platform and should support different
processors. 
 
\subsection{Related Work}
There are various performance models other than the roofline model.
\cite{Asanovic:EECS-2006-183} \cite{Boyd94ahierarchical} In particular, the 3Cs
(compulsory, capacity and conflict misses) model for caches
\cite{Hill:1989:EAC:76602.76603} is simple to understand, yet offers
insight into program behavior.

Multitasking operation systems may interrupt a program at any time and
perform a context switch. When a measurement run is interrupted the result can
be affected. Techniques to deal with this problem are
presented in \cite{ComSysProgPersp}, with a focus on measuring execution time.

In addition, the initial state of the caches can have an influence on the
measurement results. It is therefore important to warm or flush the caches. An
in detail discussion can be found in \cite{Whaley:2008:AAC:1462062.1462065}.

\subsection{Contribution of the Thesis}
The main contribution of this thesis is of practical nature. We built a tool
which simplifies performance measurements and in particular the generation of
roofline plots.

The tool provides a set of tested measurement routines, which can be used
to measure custom code. Using the provided infrastructure simplifies the
analysis of the precision of the generated results.

We presented detailed low level measurement results and analyzed their accuracy
and precision. This could be used as reference for future work.

We implemented a variety of kernels (in many cases using libraries) and
discussed them in the context of the roofline model. 

Finally, we extended the roofline plots in the following ways: 
\begin{description}
\item[Series] We connected multiple points of a measurement series
(for example multiple problem sizes of a single algorithm implementation) with a line.
\item[Series Comparisons] When measuring series of different
implementations of a single algorithm, we connected equal problem sizes with a
dashed line to faciliates the comparison between the implementations.
\item[Error Boxes] We repeated our measurements and represented the precision
with error boxes within the roofline plots.
\end{description}

\subsection{Acknowledgements}
First I would like thank Georg Ofenbeck for supervising my thesis. He guided me
through the process of writing this thesis. In addition he proof-read my drafts
and provided very valuable feedback.

I worked in the group of Professor Markus P\"uschel. He creates a very
positive working atmosphere among his staff. We held few but very productive
meetings. They always helped me to see clear again.

Finally I would like to thank Daniele Spampinato and Zolt\'an Maj\'o for the
intresting discussions we had.

\section{Measurement Setup}
In this section we first introduce the performance counters, which are
used to measure most of the quantities required to generate a roofline
plot. Altough the performance counters are available on all newer x86 CPUs,
they differ between processors. Therefore we describe our measurement
machine and how we measured on that particular machine.

Finally, we present our method of cache initialization and how we handle
frequency scaling.
 
\subsection{Performance Counters}
One of the key features of a CPU is it's speed. But that speed can only be
shown if the software running on the CPU is properly optimized. Since
optimization often requires detailed understanding of the CPU's internal
behavior, performance events and counters were incorporated directly into the
CPU. 

Using special control registers, each counter can be configured to increment
when a performance event occurs. Since the counters are implemented in hardware,
enabling them does not have any effect on program execution (no overhead). 

Recent x86 CPUs typically contain two performance counters and support a large
number of performance events, 119 on a CoreDuo. Examples are:
\begin{description}
  \item[INSTRUCTION RETIRED] occurs for every retired instruction
  \item[DBUS BUSY] occurs for every cycle during which data bus is busy
  \item[SSE PRE MISS] occurs when an SSE instruction misses all cache levels
  \item[BR TAKEN RET] occurs for every retired taken branch instruction 
\end{description}

When working with the linux operation system, the performance counters are
configured and read using a kernel API. The kernel takes care of isolating
different processes, such that each process can configure and use the counters
according to the specific requirements, without interfering with other processes.

Using the performance counters, it is possible to measure execution time, memory
traffic and operation count. For the execution time either the cycle counter or
the system timer can be used. The memory traffic can be determined using the
performance counters for counting cache misses and write back operations. An
other option is to use the counters for bus transfers. Measuring the operation count
depends heavily on the definition of operation, but is typically measurable with
the performance counters, too.

\subsection{Measurement Machine}
\label{sec:MeasurementMachine}
All measurements are performed on an IBM X60 Thinkpad, featuring an Intel
CoreDuo CPU (Family 6, Model 14). The CPU is based on the Yonah
micro architecture, which is similar to the Pentium-M. The CPU contains two
cores, each having a 32KB instruction and a 32KB data L1 cache, 8 ways set associative,
with 64 bytes line size. The two cores share a 2MB unified L2 cache, again 8
ways set associative and with 64 bytes line size. The core frequency can scale
between 1GHz and 1.83GHz. The bus frequency is 167MHz. 

The main memory consists of two 2GB DDR2 modules, totaling in 4GB available
memory. The theoretical throughput of the memory is 5.12 GB/s, which is 2.80
Bytes per core cycle, if the CPU runs at 1.83GHz.

We used XUbuntu 11.10, running a Linux 3.0.0-16 kernel in 32 bit mode, since
the CPU does not support the 64 bit mode. We used GCC 4.6.1.

The performance counters are instrumented using the 'perf event' kernel
interface \cite{unoffPerfEventsWebPage}, using libpfm4  \cite{libpfm4Docu} to
generate the required parameters.

We used 'coreduo::UNHALTED\_CORE\_CYCLES' for measuring time. For the operation
count, we used the follwing definitions for operation: 
\begin{description}
\item[SinglePrecisionFlop] SSE single precision operations.
\\{\footnotesize
'coreduo::SSE\_COMP\_INSTRUCTIONS\_RETIRED:SCALAR\_SINGLE'\\
+2*'coreduo::SSE\_COMP\_INSTRUCTIONS\_RETIRED:PACKED\_SINGLE'}
\item[DoublePrecisionFlop] SSE double precision operations.
\\{\footnotesize
'coreduo::SSE\_COMP\_INSTRUCTIONS\_RETIRED:SCALAR\_DOUBLE'\\
+2*'coreduo::SSE\_COMP\_INSTRUCTIONS\_RETIRED:PACKED\_DOUBLE'}
\item[CompInstr] Computational instructions retired. Counts SSE instructions
and x87 instructions. Used for x87 code. 
\\{\footnotesize 'coreduo::FP\_COMP\_INSTR\_RET'}
\item[SSEFlop] SSE operations, sum of SinglePrecisionFlop and
DoublePrecisionFlop
\end{description}

We used two variants of measuring the memory transfer volume. The {\bf MemBus}
variant uses 64*'coreduo::BUS\_TRANS\_MEM', which measures the transfers on the
system bus. The {\bf MemL2} variant uses the counters for the L2 cache line
allocation and eviction, namely
64*('coreduo::L2\_LINES\_IN:SELF'+'coreduo::L2\_M\_LINES\_OUT:SELF'), combined
with 8*'coreduo::SSE\_NTSTORES\_RET' to take non temporal stores into account.

\subsection{Cache}
\graphFloat{yonah/FFTwarm}{Influence of the initial cache state (MemL2)}
\graphFloat{yonah/FFTwarmPerformance}{Influence of the initial cache
state on the Performance} 

Specially for short running kernels, the initial state of the cache can have a
big impact on the memory traffic and execution time.  The impact can be
controlled by making sure the caches are either warm or cold.

In our tool, it is possible to specify the desired initial state of the code and
the data separately. We used the fast fourier transformation implementation of
the Intel Math Kernel Libraries as showcase. Figures \ref{yonah/FFTwarm} and
\ref{yonah/FFTwarmPerformance} show the results when the cache is cold
(FFT-Mkl), only the data is warmed up (FFT-Mkl-Data), only the code is warmed up
(FFT-Mkl-Code) or both data and code is warmed up (FFT-Mkl-Data-Code).

In the roofline plot (fig. \ref{yonah/FFTwarm}) we repeated each measurement
10 times and showed the median result, as well as the 25\% and 75\% percentile
as a box around the median point. We connected the median points for the same
problem size with dashed lines.

If only the code cache is warmed up, the difference to completely cold caches
is small. The effect of warming up the data chaches more significant. 

If the data and the code is loaded into the caches initially, the operational
intensity increases again compared to only loading the data.  Short before the
cache size is exceeded by the size of the data and the code, the operational
intensity becomes very large.

For large problem sizes, the initial state of the cache has a neglectable
impact.

To get a better impression on the effects on performance, figure 
\ref{yonah/FFTwarmPerformance} shows the problem size on the x axis and the
performance on the y axis. For a problem size of 10'000, the performance
difference between cold and warm caches (code and data) is around 20\%.


The following snippet shows the relvant logic for initializing the caches:

\begin{minipage}{\textwidth-18pt}
\begin{lstlisting}[language=C++]
// should the code cache be warm?
if (getWarmCode()) {
	// Tell the kernel to warm the code cache,  
	// which usually results in the kernel 
	// beeing executed once.
	getKernel()->warmCodeCache(); 
	
	// Should we clear the data?
	if (!getWarmData()) {
		getKernel()->flushBuffers();
	}
} else { // Code cache should be cold.
	// Access a large memory buffer and execute  
	// a lot of code, this clears the code cache.
	clearCaches();
	
	// Should the data be warm?
	if (getWarmData()) {
		// Warm the data cache by accessing each 
		// cache line of the data buffer(s).
		getKernel()->warmDataCache();
	} else {
		// The data cache should be cold. 
		getKernel()->flushBuffers();
	}
} 
\end{lstlisting}
\end{minipage}

Getting cold caches can be quite tricky. See
\cite{Whaley:2008:AAC:1462062.1462065} section '3.
CACHE FLUSHING METHODS WHEN TIMING ONE INVOCATION' for details.

To clear the code cache, we use the traditional method of accessing a large
memory buffer and executing more than 32KB code to flush the L1 code cache.

In addition, the kernels report the buffers they allocated. This is used to
flush these buffers with the 'clflush' instruction, not affecting code which is
already in the cache. This is implemented by the \method{flushBuffers()}
method in the above code snippet.

\subsection{Frequency Scaling}
The core frequency is not constant in current processors. Since the memory
latencies and throughputs do not scale with the core frequency, a lower core
frequency will generally cause a higher percentage of the peak performance to be
reached by the kernel.

On Linux, frequency scaling is controlled so called governors. We used the
'performance' governor for all our measurements, which fixes the core frequency
to the maximum.

\section{Single Threaded Accuracy and Precision}
First, let's recapitulate the definitions of accuracy and precision.
\cite{accuracyAndPrecision} Accuracy of a measurement system is the closeness of
the measurement results to the true value.
Precision is the degree to which repeated measurements show the same result.

A standard technique to increase precision is to repeat a measurement and use
a measure of tendency. This generally increases the precision of the result, but
does not improve the accuracy in presence of systematic errors.

To analyze the results of single threaded measurements we designed three groups
of measurements. Each group is focused on one specific quantity, namely time,
transfer volume and operation count. We measured on an idle system. All
measurements are executed with cold caches.

In all groups we used the following kernels:
\begin{description}
\item[ADD] Repeat adding a constant to an accumulator. Everything is performed
in registers. We use multiple accumulators and unroll the loop. 
\item[MUL] Repeat multiplying an accumulator with a constant. Everything is performed
in registers. We use multiple accumulators and unroll the loop. 
\item[Read] Read a buffer from memory
\item[Write] Overwrite a buffer in memory
\item[WriteStream] Overwrite a buffer in memory, using non temporal store
instructions
\item[Triad] Perform $a_i=b_i+k*c_i$. This involves reading $a$, $b$ and $c$ and
writing $a$ back.
\end{description}

Read, write and WriteStream read or write a single memory buffer of size $S$.
The Triad kernel operates on three buffers of size $S$.
 
\subsection{Transfer Volume}

\graphFloatTwo{Ratio of the Actual Transfer Volume to
the Expected Transfer Volume} {yonah/valTBValues}{MemBus}
{yonah/valTBALTValues}{MemL2}

We tried to estimate the amount of memory $\Theta$ that has to be transferred. 
For pure reading, we expect to observe the whole buffer being transferred to the
core ($\Theta=S$).
When writing is involved, we expect some of the writes to be held back in the
cache. For the write kernel the buffer is loaded into the CPU before beeing
overwritten. In addition we presume that the whole cache is used for write back
caching. Thus the estimated transfer volume is $\Theta=S+\max(0,S-2MB)$. 

For the triad kernel, all three buffers have to be loaded into the CPU and one
has to be written back. We presume that one third (due to the three buffers) of
the cache is used for write back caching. ($\Theta=3S+\max(0,S-\frac23MB)$) The
streaming write kernel should write the whole buffer once ($\Theta=S$).

In figure \ref{yonah/valTBValues} we show the expected transfer volume $\Theta$
on the $x$ axis and the ratio of the actual and the expected transfer volume on
the $y$ axis ($\frac{\tau}{\Theta}$). We measured system bus transfers in
subfigure {\bf a} (MemBus) and L2 cache line allocations in subfigure {\bf b}
(MemL2).

The distribution of the results is shown using box plots. \cite{BoxPlot} For
each expected transfer volume, the measurement is repeated $n=100$ times. The
median is shown with a point and connected to the medians of the other transfer
volumes with a line. Around the point a box is plotted. The top of the box
represents the 75th percentile, the bottom the 25th percentile. The ends of the
whiskers represent the minimum and maximum of the observed results. 

With our measurement setup, we roughly observe the expected transfer volumes.
Around an expected transfer volume of 2MB we overestimate the influence of the
write back cache. The CPU seems to write data back to the memory before the
cache is completely filled.

The MemBus and MemL2 variants produce similar result. Note that the MemL2
variant measures less than the expected memory transfer for large buffer sizes.

To show the error of the measurement results we defined the $\err(x)$ function: 

$$
\err(x)=\begin{cases}
x>1 & (x-1)*100\%\\
x\leq1 & (\frac{1}{x}-1)*100\%
\end{cases}
$$

\graphFloatTwo{Error of the Transfer Volume: $\err(\frac{\tau}{\Theta})$}
{yonah/valTBError}{MemBus}
{yonah/valTBALTError}{MemL2}
We used the $\err(x)$ function for the $y$ axis of figure
(\ref{yonah/valTBError}). The spread of the MemL2 variant is better than the
spread of the MemBus variant, while the offset is comparable.

\graphFloat{yonah/valTBALTFlushValues}{MemL2: Memory Transfer during Cache
Flush after kernel execution}

To validate our assumption on the write back cache, we measured the memory
transfer while flushing the cache using 'CLFLUSH' after the execution of the
kernel completes. (fig. \ref{yonah/valTBALTFlushValues}) 

There is a small overhead for the read kernel. For the write and the triad
kernel, all written data is transferred while flushing the caches as long as
the data fits into the cache. 

If the data does not fit into the cache anymore, almost the whole 2MB 2nd level
cache is used for write back caching in case of the write kernel. For the Triad
kernel one could suspect that each buffer receives about one third of the cache,
which is affirmed by the data.

When measuring on an Intel Core, we observed a strange behavior. While the Read
and the Triad kernels behaved as expected, the we observed a memory transfer of
half the buffer size for the write kernel, independent of the buffer size. We
used the SSE intrinsics. Switching to the normal integer registers produced the
expected results.

\graphFloatTwo{Transferred Bytes of the Arithmetic Kernels}
{yonah/valTBArithTBValues}{MemBus}
{yonah/valTBALTArithTBValues}{MemL2}

When measuring the memory transfer of the arithmetic kernels, we observe an
overhead of around 2.5KB for for short running kernels. When the execution time
exceeds 1ms to 10ms, the memory transfer starts to correlate with the execution
time. There is a huge difference between the BusMem and BusL2 variants. The
BusMem variant measures an overhed of around one byte every 20 cycles, while the
BusL2 variant measures one byte every 300-1600 cycles (see fig.
\ref{yonah/valTBArithTBValues} and \ref{yonah/valTimeArithValues}) This
difference is plausible since the BusMem variant is likely to pick up any noise
on the memory bus.

\clearpage

\subsection{Time}
\graphFloat{yonah/valTimeArithValues}{Execution Time of the Arithmetic Kernels}
\graphFloat{yonah/valTimeArithError}{Error of the Execution Time of the
Arithmetic Kernels: $\err\left(\frac{\text{Execution Time}}{\min(\text{Execution
Time})}\right)$}

We analyzed the execution time of the arithmetic ADD and MUL kernels. Both
should theoretically cause no memory transfer. Figure
\ref{yonah/valTimeArithValues} shows the execution time of the kernels for
different iteration counts. The execution time is one cycle per operation for
the ADD kernel and two cycles per operation for the MUL kernel. The difference
between using the legacy x87 floating point instructions and SSE is neglectable.
These results are consistent with the information found in the Intel manual.

For the error plot we use the minimal execution time for each expected operation
count as reference.  The error is small for operation counts above 1'000'000
(fig \ref{yonah/valTimeArithError}).

\graphFloat{yonah/valTimeMemValues}{Execution Time of the memory kernels} 
\graphFloat{yonah/valTimeMemError}{Error of the Execution Time of the Memory
Intensive Kernels:  $\err\left(\frac{\text{Execution Time}}{\min(\text{Execution
Time})}\right)$} 

For the kernels causing memory transfer, we observe a
strong correlation between expected memory transfer and execution time (fig. \ref{yonah/valTimeMemValues}), but the errors far exceed those observed for the arithmetic kernels (fig.
\ref{yonah/valTimeMemError}).

\clearpage
\subsection{Operation Count}
\graphFloat{yonah/valOpValues}{Operation Count: $\frac{\text{Actual Operation
Count}}{\text{Expected Operation Count}}$} Figure \ref{yonah/valOpValues} shows
that we get almost perfect results for measuring the operation count.

\clearpage
\subsection{Handling the Variance}
\label{sec:SingleThreadedHandlingTheVariance}
In the previous sections we saw that our measurements produce plausible results.
As soon as the kernels cause memory transfer, we observe a high variance. We
presume this is due to various activities happening in the background during the
measurement. (Task switches etc).

These activities cause events to be measured which are not related to
the execution of the kernel. Fortunately, additional events only increase the
measured execution time, memory transfer or operation count. Ideally, at least
one execution does not include any overhead. In this case the true value is the
minimum value observed.

Based on these considerations, we repeat each measurement $K=10$ times and
discard all but the minimum value. Results generated using this scheme are
marked by adding value of $K$ as subscript.

\graphFloatTwo{Error of the Transfer Volume with $K=10$:
$\err\left(\frac{\tau_{10}}{\Theta}\right)$} {yonah/valTBMinError}{MemBus}
{yonah/valTBALTMinError}{MemL2} 

Using this scheme for measuring the transfer volumes, the samples
fall in a range of about 5\%. The bias is under 10\% except for transfer volumes
under 20KB and with a peak of up to 40\% around 2MB for the kernels involving
writes. The differece in precision between MemBus and MemL2 disappeared
almost completely. For large transfer volumes, the accuracy of the MemBus
variant is superior. (fig. \ref{yonah/valTBMinError})

\graphFloat{yonah/valTimeMemMinError}{Error of the Execution Time with
$K=10$: $\err\left(\frac{\text{Execution
Time}_{10}}{\min(\text{Execution Time})}\right)$}

Applying the same scheme to measuring the execution time of the memory kernels
yields a precision of about 10\%. Since we cannot derive an expected value, we
chose the minimum observed value as reference.
(fig. \ref{yonah/valTimeMemMinError})

\graphFloat{yonah/valTimeArithMinError}{Error of Execution Time with
$K=10$: $\err\left(\frac{\text{Execution
Time}_{10}}{\min(\text{Execution Time})}\right)$}

\graphFloat{yonah/valOpMinError}{Error of Operation Count with
$K=10$: $\err(\frac{\text{Actual Operation Count}_{10}}{\text{Expected
Operation Count}})$}

Finally, figures \ref{yonah/valTimeArithMinError} and \ref{yonah/valOpMinError}
show that the arithmetic kernels do not pose any problems.

Summarizing our results, using the minimum of $K$ scheme with $K=10$ the
arithmetic kernels can be measured with high precision and accuracy. The results
of measuring execution time and memory transfer of memory intensive kernels are
scattered within 10\% of the reference value. The memory tranfer volume is
overestimated by about 10\% for transfer volumes under 10MB.

\clearpage
\section{Multi Threaded Accuracy and Precision}
For the validation of multi threaded measurements, we use the same measurements
as for the single threaded validation, but run a separate workload instance on
each core. Using a barrier, we make sure the two kernels start at the same time. 

Since the Yonah architecture does not utilize HyperThreading, we don't expect
an arithmetic kernel running on one core affecting a memory intensive kernel on
the other core. Therefore the workloads run in parallel are not mixed.

\subsection{Transferred Volume}
\graphFloatTwo{Ratio of the Actual Transfer Volume to the Expected Transfer
Volume for two Read Kernels running in Parallel (using one Buffer per Kernel)}
{yonah/valTBThReadValues}{MemBus} {yonah/valTBALTThReadValues}{MemL2} 

\graphFloatTwo{Ratio of the Actual Transfer Volume to the Expected Transfer
Volume for two Write Kernels running in Parallel (using one Buffer per Kernel)}
{yonah/valTBThWriteValues}{MemBus}
{yonah/valTBALTThWriteValues}{MemL2} 

\graphFloatTwo{Ratio of the Actual Transfer Volume to the Expected Transfer
Volume for two Write Kernels using Streaming Stores running in Parallel (using
one Buffer per Kernel)} 
{yonah/valTBThWriteStreamValues}{MemBus}
{yonah/valTBALTThWriteStreamValues}{MemL2}

\graphFloatTwo{Ratio of the Actual Transfer Volume to the Expected Transfer
Volume for two Triad Kernels running in Parallel (using one Buffer per Kernel)}
{yonah/valTBThTriadValues}{MemBus}
{yonah/valTBALTThTriadValues}{MemL2} 

\graphFloat{yonah/valTBThReadPoint}{Distribution of the Results of the Read
Kernel using MemBus} 

\graphFloat{yonah/valTBThWritePoint}{Distribution of the Results of the Write
Kernel using MemBus}

\graphFloat{yonah/valTBThWriteStreamPoint}{Distribution of the Results of the
Write Kernel using Streaming Stores measured with the MemBus measurement
variant}

\graphFloat{yonah/valTBThTriadPoint}{Distribution of the Results of the Triad
Kernel using MemBus} 

The MemBus variant measures all transfers on the bus. Therefore, the
measurer of each workload should see the traffic of both kernels. The MemL2 variant
should separate the traffic of the two cores. On figures
\ref{yonah/valTBThReadValues},
\ref{yonah/valTBThWriteValues},
\ref{yonah/valTBThWriteStreamValues} and
\ref{yonah/valTBThTriadValues} we see the expected results for the MemL2
variant. The precision is comparable to the single threaded measurements, the
overhead is slightly higher.

In contrast, the MemBus variant shows a huge variation. Choosing 4MB as
representative buffer size, figures \ref{yonah/valTBThReadPoint},
\ref{yonah/valTBThWritePoint},
\ref{yonah/valTBThWriteStreamPoint} and \ref{yonah/valTBThTriadPoint} show the
distribution of the results. There is a cluster around the expected result of
twice the expected transfer volume (each workload seeing the transfer of both
cores), but also a cluster around once the expected transfer volume. It seems
that in about half of the measurement runs, the kernel only see their own
traffic. 

\subsection{Time}
\graphFloat{yonah/valTimeThAddArithValues}{Execution Time of two ADD Kernels
running at the same time} As expected, running two arithmetic kernels in
parallel does not affect the measured execution time (fig. \ref{yonah/valTimeThAddArithValues}).

\graphFloat{yonah/valTimeThReadMemValues}{Execution Time of two Read Kernels
running at the same time}

\graphFloat{yonah/valTimeThWriteMemValues}{Execution Time of two Write Kernels
running at the same time}

\graphFloat{yonah/valTimeThTriadMemValues}{Execution Time of two Triad Kernels
running at the same time}

For the memory kernels we don't see much difference in execution time between
the single threaded and the multi threaded results for expected transfer volumes
below 1MB. We expected to observe a larger difference. For larger buffers the
expected difference of about a factor of two can be observed, although outliers
are frequent. (fig. \ref{yonah/valTimeThReadMemValues} and
\ref{yonah/valTimeThWriteMemValues})

For the triad kernel there is almost no
difference. (fig. \ref{yonah/valTimeThTriadMemValues})

\subsection{Operation Count}

\graphFloat{yonah/valOpThValues}{Ratio of the Actual Operation Count to
the Expected Operation Count for two ADD Kernels running at the same time}

As for the single threaded case, measuring the operation count does not pose any
problems. (fig. \ref{yonah/valOpThValues})

\clearpage
\subsection{Handling the Variance}
\label{sec:MultiThreadedHandlingTheVariance}
We again use the minimum of $K=10$ scheme. (see
\ref{sec:SingleThreadedHandlingTheVariance})

\graphFloatTwo{Error of the Transfer Volume of two Read Kernels running at the
same time with $K=10$: $\err\left(\frac{\tau_{10}}{\Theta}\right)$}
{yonah/valTBThReadMinError}{MemBus} {yonah/valTBALTThReadMinError}{MemL2} 

\graphFloatTwo{Error of the Transfer Volume of two Write Kernels running at the
same time with $K=10$: $\err\left(\frac{\tau_{10}}{\Theta}\right)$}
{yonah/valTBThWriteMinError}{MemBus} {yonah/valTBALTThWriteMinError}{MemL2} 

\graphFloatTwo{Error of the Transfer Volume of two Write Kernels using
Streaming Stores running at the same time with $K=10$:
$\err\left(\frac{\tau_{10}}{\Theta}\right)$}
{yonah/valTBThWriteStreamMinError}{MemBus}
{yonah/valTBALTThWriteStreamMinError}{MemL2}

\graphFloatTwo{Error of the Transfer Volume of two Triad Kernels running at the
same time with $K=10$: $\err\left(\frac{\tau_{10}}{\Theta}\right)$}
{yonah/valTBThTriadMinError}{MemBus} {yonah/valTBALTThTriadMinError}{MemL2} 

When measuring the transfer volume we achieve a variation below 10\%. The
overhead is around 20\% for small buffers, and goes towards zero for transfer
volumes exceeding 10MB. (figs \ref{yonah/valTBThReadMinError},
\ref{yonah/valTBThWriteMinError},
\ref{yonah/valTBThWriteStreamMinError} and \ref{yonah/valTBThTriadMinError})

\graphFloat{yonah/valTimeThReadMemMinError}{Error of the Execution Time of two
Read Kernels running at the same time with $K=10$: $\err\left(\frac{\text{Execution
Time}_{10}}{\min(\text{Execution Time})}\right)$}

\graphFloat{yonah/valTimeThWriteMemMinError}{Error of the Execution Time of two
Write Kernels running at the same time with $K=10$: $\err\left(\frac{\text{Execution
Time}_{10}}{\min(\text{Execution Time})}\right)$}

\graphFloat{yonah/valTimeThWriteStreamMemMinError}{Error of the Execution Time
of two Write Kernels running at the same time with $K=10$: $\err\left(\frac{\text{Execution
Time}_{10}}{\min(\text{Execution Time})}\right)$}

\graphFloat{yonah/valTimeThTriadMemMinError}{Error of the Execution Time of two
Triad Kernels running at the same time with $K=10$: $\err\left(\frac{\text{Execution
Time}_{10}}{\min(\text{Execution Time})}\right)$}

For the execution time, we observe huge variations for small buffer sizes. If
the expected transfer volume exceeds 20KB, the variations drop to around 10\%
and become large again for transfer volumes exceeding 10MB. (figs.
\ref{yonah/valTimeThReadMemMinError}, \ref{yonah/valTimeThWriteMemMinError},
\ref{yonah/valTimeThWriteStreamMemMinError} and
\ref{yonah/valTimeThTriadMemMinError})

\graphFloat{yonah/valTimeThAddArithMinError}{Error of the Execution Time of two
ADD Kernels running at the same time with $K=10$: $\err\left(\frac{\text{Execution
Time}_{10}}{\min(\text{Execution Time})}\right)$}

The variations observed when running two ADD kernels in parallel are small
except for iteration counts below 100000. (fig.
\ref{yonah/valTimeThAddArithMinError})

\graphFloat{yonah/valOpThMinError}{Error of the Operation Count of two ADD
Kernels running at the same time with $K=10$:
$\err(\frac{\text{Actual Operation Count}_{10}}{\text{Expected Operation Count}})$}
As expected, the results for the operation count are perfect (fig. \ref{yonah/valOpThMinError})

Summarizing our results, using the minimum of $K$ scheme with $K=10$ the
arithmetic kernels can be measured with high precision and accuracy. 

The results of measuring the transfer volume of memory intensive kernels are
scattered within 10\% of the reference value and overestimated by about 20\% for
transfer volumes below 10MB. The results of measuring the execution time have
generally a high variance, except for transfer volumes between 20KB and 10MB.

\clearpage

\section{Finding Performance and Bandwidth Ceilings}
Roofline plots show ceilings for the performance and the bandwith of the
measurement system. The values for the ceilings can be derived from the system
specification. 

For the Yonah architecture, one double precision addition can be issued every
cycle and one multiplication every two cycles. The latencies are 3 cycles for
the addition and 5 cycles for the multiplication. \cite{inteloptimize} 

The latency specifies how many cycles after issuing an operation the result
becomes available. If the result of a first operation used as input for a second
operation, the second operation has to wait for the first operation to complete.

There is no difference between the x87 and the SSE instruction set, since the
same execution units are used for both. The performance of SSE code can be
slightly better, since more registers are available and fewer instructions need
to be decoded.

The theoretical ceiling for pure additions is 1 flop/cycle and 0.5 flop/cycle
for pure multiplications.

It should be possible to use the multiplier and the adder at the same time, so
if there are two additions per multiplication it should be possible to reach 1.5
flop/cycle.

We implemented a heavily optimized micro benchmark for pure adds, pure
multiplications and a mix of additions and multiplications. For the additions we
used the following loop:

\begin{lstlisting}
double r[DLP];
...
for (long i = 0; i < iterations; i++) {
	for (int p = 0; p < UNROLL; p++) {
		for (int j = 0; j < DLP; j++) {
			r[j] += 1;
		}
	}
}
\end{lstlisting}

We allocated an array of DLP doubles. In each iteration, we add 1 to each of
the doubles in the array. This avoids the latency and keeps multiple
instructions in flight. For the SSE implementation we did a straight
forward vectorization of the code using SSE intrinsics.

For the multiplication we use a very similar loop. We just replaced the
addition with a multiplication. 

\begin{lstlisting}
double r[DLP];
...
for (long i = 0; i < iterations; i++) {
	for (int p = 0; p < UNROLL; p++) {
		for (int j = 0; j < DLP; j++) {
			r[j] *= base;
		}
}
}
\end{lstlisting}

Finally, we created a mix with two additions per multiplication:

\begin{lstlisting}
double mulR[DLP];
double addR[DLP*2];
...
for (long i = 0; i < iterations; i++) {
	for (int p = 0; p < UNROLL; p++) {
		for (int j = 0; j < DLP; j++) {
			mulR[j]*=base;
			addR[j]+=1;
			addR[DLP+j]+=1;
		}
	}
}
\end{lstlisting}

We optimized the value for UNROLL and DLP in the range of 1 to 19 and found the
following results:

\begin{center}
\begin{tabular}{llllr@{.}lr@{.}l}
\toprule
Operation&Instruction Set & DLP & UNROLL &  \multicolumn{4}{l}{Performance
[Flop/Cycle]} \\ \cmidrule{5-8}
&&&&\multicolumn{2}{l}{Measured}&\multicolumn{2}{l}{Theoretical}\\
\midrule
ADD   & x87       &  9 & 14 & 0&965 & 1&0\\
      & SSEScalar &  7 & 15 & 0&934\\
      & SSE       &  7 & 15 & 0&986\\
MUL   & x87       &  3 & 10 & 0&498 & 0&5\\
      & SSEScalar &  3 &  9 & 0&498\\
      & SSE       &  3 & 10 & 0&498\\
Mixed & x87       &  3 & 17 & 1&31 & 1&5\\
      &SSEScalar  &  3 & 17 & 1&35\\
      &SSE        &  2 &  4 & 1&34\\
\bottomrule
\end{tabular}
\end{center}

For pure additions and multiplications we almost reached the theoretical limit.
For the mixed workload we reached 90\% of the theoretical 1.5 flops per cycle.

In our plots we show 1 flop per cycle as single core performance ceiling and 2
flops per cycle as dual core performance ceiling.

The bandwidth ceiling can be calculated from the parameters of the memory
subsystem and is 2.80 bytes per cycle (see section
\ref{sec:MeasurementMachine}).

To measure the bandwidth which can actually be reached we implemented a
microbenchmark. To measure the read performance, the benchmark iterates over a
buffer and combines the data using the XOR operation. For the write performance
a memory buffer is overwritten.

We use SSE load and stores. There are two flavors of the store operations.
Normal stores first load the data from memory into the caches and overwrite it
there. When the cache line gets evicted, the data is written to memory.
Non temporal stores do not load the data into the caches. The data is written
directly to memory.

We use multiple load and store streams. That is, we split the buffer
into multiple parts and iterate over all parts simultaneously.

We experimented with software prefetching but could not improve our results. 

Optimizing loop unrolling and stream count resulted in  the following
numbers:

\begin{center}
\begin{tabular}{lllr@{.}l}
\toprule
Operation & Streams & UNROLL &  \multicolumn{2}{l}{Throughput
[Byte/Cycle]} \\
\midrule
Load  &  2 & 3 & 2&16\\
Write &  3 & 4 & 1&71\\
Non Temporal & 1 & 1 & 1&66\\
\bottomrule
\end{tabular}
\end{center}

The throughput of the non temporal stores seems to be low compared to the
normal stores. But since only half of the memory has to be transferred (no
loads), the same amount of memory is actually written in half the time.

The results reported here were obtained using MemBus, but the results of the
MemL2 variant are almost identical.
 
In the stream benchmark \cite{stream} the system reaches a copy rate of 2434
MB/s, which is $2*3434MB/1.8GHz=2.84$ bytes per cycle. Our microbenchmark
reported a smaller number, but the result is comparable to ours.

We also measure the random access throughput. A large buffer is allocated and
randomly accessed. The troughput is 0.525 bytes per cycle. Since a cache line is
64 bytes, this results in one memory access every $64/0.525=122$ cycles.

In our roofline plots we only show the load ceiling (2.2 flops per cycle) and
the random access ceiling (0.52 bytes per cycle).

\section{Experimental Results}
We used our tool to measure various kernels. For each kernel we describe the
kernel and the measurement setup and show the resulting roofline plot.

We used the techniques described in sections
\ref{sec:SingleThreadedHandlingTheVariance} and
\ref{sec:MultiThreadedHandlingTheVariance} to reduce the variance of the
measurement results. 

We repeated each measurement 10 times and show the results using whisker bars
for the minimum and maximum and error boxes for the 25th and 75th percentile. 

Since the precision generally improves with growing problem sizes, we decided to
measure large problems only once. For such measurements, no whisker bars or
error boxes are drawn.

To investigate the influence of the initial cache state, we measured a specific
implementation of a kernel with cold caches, with the data loaded into the
caches (``-Data''), with the code loaded into the caches (``-Code'') and with
both data and code loaded into the caches (``-Data-Code'').

We observed a similar behavior for all kernels. If only the code is initially
loaded into the caches, the operational intensity is slightly increased for
small problem sizes, since the transfer volume is reduced by the size of the
code. For larger problem sizes the effect disappears, since the size of
the code is neglectable to the transfer volume needed to load the data.

If the data is initially loaded into the caches, the operational intensity is
increased compared to cold caches and grows with the size of the vector.
During the measurment the code has to be loaded once from memory.
Since the data is already in the caches the transfer volume is constant, while
the operation count grows with the problem size.

With growing problem sizes the data is too large to fit into the caches.
The operational intensity drops to the level of cold caches. The drop should
occur when the size of the data exceeds the size of the caches.

If both data and code are initially loaded into the caches, the operational
intensity is theoretically infinite (or undefined) as long as both data and
code fit into the caches, since no memory transfer is required. With growing
problem sizes the data does not fit into the caches anymore and the operational
intensity drops but stays higher than the operational intensity measured when
only the data is loaded into the caches.

For high operational intensities we observed a reduced precision. Since such
measurements run for a relatively long period of time while causing a small
transfer volume, auxiliary data transfers might be measured which occur at a
fluctuating rate. This could cause the loss of precision.

The effect on performance of the initial cache state is generally smaller than
the effect on operational intensity. While a factor of 10 is not uncommon for
the operational intensity, an increase by 1.5 can only be observed for memory
bound kernels. The effect is marginal for computationally bound kernels.

\subsection{BLAS}
Basic Linear Algebra Subprograms (BLAS) is a standard API for basic linear
algebra operations. The functionality is divided into three levels:

\begin{description}
\item[Level 1] Vector-Vector Operations 
\item[Level 2] Matrix-Vector Operations 
\item[Level 3] Matrix-Matrix Operations 
\end{description}

For each level, we chose a representative operation and generated roofline plots
using the Intel Math Kernel Library (MKL) \cite{MKL} and the OpenBlas
\cite{OpenBlas} implementations.

\subsubsection{Level 1}
\graphFloat{yonah/daxpy}{Roofline Plot of the Vector-Vector Multiplication
(MemL2, DoublePrecisionFlop)} 
For level 1, we chose the 'daxpy' operation, defined as $\bf y \leftarrow \alpha
\bf x + \bf y$ with double vectors. There is no reuse in this kernel. For every
vector element triple, two operations are performed. Since the vector sizes
we measure fits into the last level cache, we don't have to take the write
backs into account. The maximal operational intensity is $2/(2*8)=1/8=0.125$
which is confirmed by figure \ref{yonah/daxpy}. The precision of the results is
good.

\graphFloat{yonah/daxpyWarm}{Roofline Plot of the influence of the
initial cache state on the Vector-Vector Multiplication (MemL2,
DoublePrecisionFlop)}

If data or code and data is initially loaded into the caches, we expect the
operational intensity to drop for a vector size of $n=2MB/(2*8)=131'072$. But
the drop already starts at a size of 16'000 (fig. \ref{yonah/daxpyWarm}). The
two double vectors of this size occupy one eight of the cache. Since the L2
cache is 8 way associative, this is exactly the size wich fits into the cache
without using one cache line set twice. We did not investigate this further.

\subsubsection{Level 2}
\graphFloat{yonah/dgemv}{Roofline Plot of the Matrix-Vector Multiplication
(MemL2, DoublePrecisionFlop)} 
For level 2, we chose the 'dgemv' operation,
defined as $\mathbf y \leftarrow \alpha A \mathbf x  + \beta \bf{y}$ with double
vectors and matrices. For an $n\times n$ matrix and vectors of size $n$, the
operation count is $2n$ for the vector scaling and $2n^2$ for the multiplication of the matrix with the vecor and the
vector addition, totaling in $2n^2+2n$. The vector $\mathbf x$ can be reused.
Since the vector sizes we measure fit into the last level cache, the write backs
are cached. The memory transfer is $2n*8$ bytes to load the vectors and $n^2*8$
bytes to load the matrix. This results in a maximal operational intensity for
large matrices of $\approx 2/8=0.25$ which is confirmed by figure
\ref{yonah/dgemv}.

The precision is generally good, although we observe some performance outliers
for the multi threaded OpenBlas implementation.

\graphFloat{yonah/dgemvWarm}{Roofline Plot of the Matrix-Vector Multiplication
(MemL2, DoublePrecisionFlop)}

When initially loading data or code and data into the caches, the operational
intensity starts to drop at a vector size of $n=300$. At this point the matrix occupies
$300^2*8=720000$ bytes, which is about one third of the cache. 


\subsubsection{Level 3}
For level 3, we chose the 'dgemm' operation,
defined as $C \leftarrow \alpha A B + \beta C $ with double matrices. For an
$n\times n$ matrix, the operation count is $2n^2$ for the two scale operations
and $2n^3$ for the matrix-matrix multiplication and the addition to $C$.

The minimal memory transfer volume is $3n^2*8$ bytes to load the matrices and
$\max(n^2*8-k,0)$ bytes to store the result (write back caching), where $k$ is
the size of the last level cache. In our measurement system $k$ is 2MB.

The maximal operational intensity is $(2n^3+2n^2)/(3n^2*8+\max(n^2*8-k,0))$.
For large $n$ this becomes $(2n^3)/(4n^2*8)\approx n/16$.
This operational intensity cannot be reached since reuse is bound by the cache
size. The intensity achieved in practice depends on the algorithm. In the
following paragraphs we will analyze the triple loop and the blocked version.
To simplify matters, we will ignore the effect of code and small parts of
the working set on the cache.

The triple loop uses the following algorithm:

\begin{lstlisting}[language=C] 
for (long i = 0; i < size; i++) 
for (long j = 0; j < size; j++) 
for (long k = 0; k < size; k++) 
c[i * size + j] += a[i * size + k] * b[k * size + j]
\end{lstlisting}

If $B$ fits into the cache, all matrices have to be loaded exactly once. In this
case the transfer volume is $3n^2*8$ to load the matrices. Since the space
occupied by the reused matrix is not available for writeback caching,
$\max(n^2*8-\max(k-n^2*8,0),0)$ bytes are written. Dropping lower order terms,
the operational intensity is optimal ($n/16$). 

As soon as $B$ does not fit into the cache ($n^2*8>k;n>512$) it is loaded over
and over. In addition, this trashes the the write back cache. The resulting
memory transfer volume is $(n^3+3n^2)*8$ and, for large matrices, the
operational intensity is $1/4$.

For very large matrices, the lines of $A$ do not fit into the cache 
($n*8>k;n>262144$). 
 
The blocked version uses the following algorithm:

\begin{lstlisting}[language=C] 
for (long i = 0; i < size; i += Nb)
for (long j = 0; j < size; j += Nb)
for (long k = 0; k < size; k += Nb)

for (long id = i; id < i + Nb; id += Mu)
for (long jd = j; jd < j + Nb; jd += Nu)
for (long kd = k; kd < k + Nb; kd += Ku)

for (long kdd = kd; kdd < kd + Ku; kdd++)
for (long idd = id; idd < id + Mu; idd++)
for (long jdd = jd; jdd < jd + Nu; jdd++)

c[idd * size + jdd] += 
  a[idd * size + kdd] * b[kdd * size + jdd];
\end{lstlisting}

We chose $N_b=50$ and $N_u=M_u=K_u=2$.
For our discussion, only the blocking with $Nb$ is relevant. It is presumed that
$N_b$ divides $n$. 

If the matrix $B$ fits completely into the cache ($n^2*8<k;n<512$), we again
have a memory transfer volume of $3n^2*8+\max(n^2*8-\max(k-n^2*8,0),0)$. 

The blocking divides the matrices into block lines and block columns, each
containing $\frac{n}{N_b}$ blocks.

If the cache cannot hold the whole matrix but a whole block line fits into the
cache ($\frac{n}{N_b}N_b^2*8=nN_b*8<k;n<5243$), the blocks of $A$ are reused.
Thus $A$ is loaded once while each block of $B$ is loaded $\frac{n}{N_b}$ times,
once per block line of $A$. The memory transfers due to the loads is
$(2n^2+\frac{n}{N_b}n^2)*8=(2n^2+\frac{n^3}{N_b})*8$. The effect of outstanding
writes can be neglected. The writes cause an additional $n^2*8$ bytes to be
transferred. Dropping lower order terms, the operational intensity becomes
$2n^3/(n^3/N_b*8)=N_b/4=12.5$

For large matrices the block lines of $A$ do not fit into the cache ($n>5243$).
Each block of $A$ is loaded once per block column of $B$ and each block of $B$ is loaded
once per block row of $A$. Thus the memory transfer to load $A$ and $B$ is
$2\frac{n}{N_b}n^2*8=2\frac{n^3}{N_b}*8$ and $2*n^2*8$ to load and store $C$.
The resulting operational intensity is $N_B/8=6.25$.

\graphFloat{yonah/mmmOp}{Matrix Matrix Multiplication: Operational Intensity
(MemL2)} When comparing these models for the operational intensity of the triple
loop and the blocked version, we observe a good match for matrix sizes up to $n=200$.
Past this size, the operational intensity of the triple loop rapidly falls to
the value predicted for large matrices, but the drop begins too early. We
predicted the fall when a single matrice occupies more memory than the L2 cache
size, but the measured drop begins at half the cache size. (fig \ref{yonah/mmmOp}).

The model for the blocked version overestimates the measurement results as
well. The initial drop begins too early. The operational intensity for large
matrices is about half the predicted value. 

In attempt to find the reason for the observed behavior we analyzed the TLB
misses. The TLB has 128 entries. For the triple loop we predict
$3*\frac{n^2*8}{4KB}$ TLB misses as long as one matrix fits into 128 pages
($\frac{n^2*8}{4KB}<128;n<\sqrt{128*4KB/8}=256$). If the matrices are larger,
calculating a single element of the result requires reading one element of every
row of $B$. If the rows of $B$ are smaller than one page, multiple accessed
elements will lie in the same page. If the rows are larger, one page is accessed
per row. Thus the number of pages accessed to compute one element of the result
is $\min(n, n^2*8/KB)$. Since the TLB is trashed during accessing a column of
$B$, there are $n^2\min(n, n^2*8/KB)$ TLB misses to compute the $n^2$ result
elements, ignoring the TLB misses due to accessing $A$ and $C$.

For the blocked version, we provide an analysis for a very restricted range
only. If one line of a matrix is larger than one page ($n*8>4KB; n>512$), each
row within a block will lie in it's own page. If each block line fits within one page
($N_b*8<4KB;N_b<512$) and the TLB is trashed during each block multiplication,
there will be $3*N_b$ TLB misses per block multiplication,
$3(\frac{n}{N_b})^3N_b=3\frac{n^3}{N_b^2}$ in total.

\graphFloat{yonah/mmmOpTlb}{Matrix Matrix Multiplication: TLB Misses}
Figure \ref{yonah/mmmOpTlb} shows that the measurement results follow our
predictions.

\graphFloat{yonah/mmm}{Roofline Plot of the Matrix-Matrix Multiplication
(MemL2)} Figure \ref{yonah/mmm} shows the roofline plot of the triple loop, the blocked
version and the two libraries. 

In addition we included the blocked version with
scalar replacement. The scalar replacement has been implemented by specifying
the matrices as 'restrict' pointers to the compiler, hence the name.

The precision of all results is decent. 

The triple loop has a poor performance, even for small matrices. This is
problably due to it's lack of unrolling. For matrix sizes exceeding $n=400$ the
drop in operational intensity and performance is clearly visible.

The other single threaded implementations all have an operational intensity of
around 10. The performance is 0.5 flops/cycle for the blocked version, 0.7 for
the blocked version with scalar replacement and around 0.85 for the library
implementaions. It is interesting to see that the highly optimized library have
a very similar operational intensity and only a 20\% higher performance compared
to the rather similar blocked version with scalar replacement.

The multi threaded implementations show a speedup of factor two, with a
reduction in operational intensity. It would be interesting to know if the
reduced operational intensity is a measurement artifact or due to the
implementations.

\graphFloat{yonah/MMMwarm}{Roofline Plot of the Matrix-Matrix Multiplication
with Different Initial Cache States (MemL2)}

\graphFloat{yonah/MMMwarmInt}{Effect of the Initial Cache State on the
Operational Intensity for the Matrix-Matrix Multiplication (MemL2)}

The initial cache state has a very small influence on the performance. (fig.
\ref{yonah/MMMwarm}) This results in a very cluttered roofline plot. The
insensitivity of the performance to the initial cache state is plausible since
the matrix-matrix multiplication has a relatively high operational intensity
and is not memory bandwidth bound. 

To be able to analyze the effect of the initial cache state on the operational
intensity, we extracted the relevant information in figure
\ref{yonah/MMMwarmInt}.

For matrix sizes greater than $n=300$ the inital cache state has almost no
effect.

The effect of loading the code into the caches is neglectable for small matrix
sizes. Starting with $n=150$ the difference becomes apparent. This could be
caused by a switch of the algorithm implementation by the library.

If the data is initially loaded into the caches the operational intensity
increases by more than an order of magnitude for small matrices. The effect
grows with larger matrices. But for a size of $n=100$ the operational
intensity begins to drop. One matrice of that size occupies 80KB only. We have
no explanation for this effect. 

At $n=150$ the drop stops, with an operational intensity still two times larger
than the one for cold caches. The effect of the warm caches decays for growing
matrix sizes. 

If both data and code is loaded into the caches we observe operational
intensities in the range of 200 to 10000 for small matrix sizes. We see a
similar drop curve as when only the data is loaded into the caches. 

The huge operational intensities can be explained by the large number of
operations that are performed while theoretically not transferring any memory.
The operation count for $n=25$ is $2*n^3=31'250$ and $250'000$ for $n=50$.
 
\subsection{FFT}
Fast Fourier Transform algorithms 'are of great importance in a variety of
fields, from digital signal processing and solving partial differential
equations to algorithms for quick multiplication of large integers.'
(\cite{FastFourierTransform})

Apart from the simple implementation found in 'Numerical Recipes' (NR)
\cite{Press:2007:NRE:1403886} on page 608, we used the implementations from the MKL \cite{MKL} , FFTW \cite{FFTW}
and Spiral \cite{Spiral}. All four implementations operate on a linear buffer
containing $n$ complex numbers, each represented by two doubles. The
transformation is performed in-place.

\graphFloat{yonah/fft}{Roofline Plot of four FFT implementations (MemL2)}
Figure \ref{yonah/fft} shows the roofline plot of the four implementations. The
NR implementation performs well initially, but the operational intensity and the
performance significantly drop for large input sizes. The library
implementations have a large overhead on small input sizes, but perform a lot
better on large inputs.

Figure \ref{yonah/FFTwarm} shows the influence of the initial cache state. When
data or code and data is initially loaded into the caches, the operational
intensity drop starts for $n=4'096$. The input size is $n*2*8=64KB$.

\subsection{WHT}
The Walsh-Hadamard transformation (WHT) is related to the FFT. It's application
is mainly in the field of data encoding.

\graphFloat{yonah/wht}{Roofline Plot WHT (MemL2)}
We used the implementation of the SPIRAL project \cite{Spiral}. The algorithm
operates on a buffer of $2^n$ doubles. Since we only have one
implementation, we directly show a roofline plot with the different
initial cache states (fig. \ref{yonah/wht})

\graphFloat{yonah/whtInt}{Operational Intensity WHT (MemL2)}
Since the roofline plot is very cluttered, we show the operational
intensity versus the problem size in figure \ref{yonah/whtInt}.
For warm data or code and data, the operational intensity drops at $n=15$. The
buffer size is $2^n*8=262'144$, which is one eight of the second level cache. We
have no explanation for this effect.

\clearpage

\section{Future Work and Conclusions}
In this thesis we designed and implemented a tool which simplifies performance
measurements. We introduced multiple abstractions, which allow simple and
common measurements to be performed with ease, while beeing flexible enough to
scale to more complex situations.

Due to these abstractions it should be possible to repeat the measurements
presented in this thesis on different systems with small effort. We already
repeated the measurements on an Intel Core machine (see \ref{}). But the
behavior of different system might be insightful.

We observe substantial variations when measuring memory intensive kernels. We
have no good explanation for these effects. A thorough analysis could be very
informative. 

In various places we observed cache misses when all data should theoretically
fit into the L2 cache. We suspect that the LRU (least recently used) cache line
replacement implementation is not perfect. Detailed experiments could possibly
tell if this is true or false.

For handling the variance we presumed that a measurement either results in the
true value, or it is disturbed by some activity (task switches) and results in a
biased result. We further presumed that the distortion can only increase the
result (time, memory transfer, operation count), and thus the minimum result of
multiple measurement runs has to be the true result.

An in depth study of this model could result in means to reduce the number of
measurements to be performed and the ability to give confidence intervals for
measurement results.

\clearpage
\appendix
\section{Measurement Tool Architecture}
The tool consists of two main components: The Measuring Core, which performs the
actual measurements, and the Measurement Driver, which controls the core.

High performance code is generally written in C or C++. Therefore the core is
written in that language. To simplify development and maintenance, we tried to
keep the amount of C++ code as small as possible. Therefore the measurement
driver is written in Java.

A measurement is controlled by a measurement specific
routine (one per measurement), which iterates through all parameter points to be examined,
compiles and starts the Measuring Core and processes the results. This should
result in straight forward code for controlling the measurements and, since the
control code is written in Java, a minimal amount of new concepts has to be
learned.

During the development of measurement control routines, it is expected that many
changes do not affect the parameter points. To speed up repeated measurements
after changes to the measurement driver, the measurement results are cached.
Thus, as long as the measurement parameters are not changed, the measurement
does not need to be repeated.

The measurement tool is used to generate and display measurement results. Often,
the measurement results lead to changes to the tool itself. Thus, switches
between using the tool and developing the tool are frequent. To support these
switches, a frontend program is provided. It compiles the measurement driver and
executes it. It can be started using a shell script called "rot". The result
files of a measurement are placed in the current working directory.

\umlDiagram{ToolComponents}

\subsection{Component Collaboration}
Data transfer between the measuring core and the measurement driver is achieved
using serialized objects stored in files. Classes of these objects are
used both from C++ and Java and therefore shared entities. They
are described using XML. Source code for both languages is generated. For
details, see \ref{sec:MultiLanguageInfrastructure} 

The root class for describing a measurement is a MeasurementCommand.
It contains the number of times the measurement should be repeated, as well as
the actual Measurement. The kernels are contained within the workloads, and
rules allow to respond to various events happening during the measurement.

\umlDiagram{MeasurementCommand}

A workload describes what should be run and measured on one core. Each workload
is run within a separate thread, which is optionally pinned to a fixed core. In
this thread, the validation measurers are started, the caches are warmed up, the
additional measurers are started, followed by the main measurer. Then the kernel
is run and the measurers are stopped. 

\umlDiagram{Workload}

During the measurement, events are raised. For example: start of a workload, end
of a workload, start of a thread etc. These events are matched against the event
predicates stored in the rules. If a predicate matches, the action of the rule
is executed.

\umlDiagram{Rule}

The following diagram shows all classes describing a measurement together:

\umlDiagram{MeasurementCommandFull}

A measurement is usually repeated multiple times, to get an idea of the
distribution of the results. Each repetition is called measurement run.
In each run, the outputs of all measurers are collected. At the end of the
measurement, the core serializes the results of all runs into a single file,
which is read by the driver.

\umlDiagram{MeasurementRunOutput}

\subsection{Tour of a Measurement}
In this section, we'll look at the components specific to a measurement. To make
sure you don't get lost, here is the tour map: 
\umlDiagram{MeasurementTourMap}

First, we'll look at the kernel. It is defined in
an XML file:
\begin{lstlisting}[language=XML]
<?xml version="1.0" encoding="UTF-8"?>
<derivedClass
	xmlns:xsi="http://www.w3.org/2001/XMLSchema-instance"
	xsi:noNamespaceSchemaLocation="../shared.xsd"
	name="TriadKernel" <!-- name of the class -->
	baseType="KernelBase"
	cSuffix="Data"
	comment="Kernel performing a=b+k*d 
		on a memory buffer">
	<field  
		name="bufferSize" 
		type="long" 
		comment="The size of the buffer"/>
</derivedClass>
\end{lstlisting}

Kernels are always derived from KernelBase, hence the derivedClass element on
line 2 and the base type defined on line 6. 

The cSuffix is given as 'Data' on
line 7. This causes the generated class to be named 'TriadKernelData'. The
measuring core implements 'TriadKernel', which derives from TriadKernelData. The
serialization service will instantiate the derived class. This mechanism allows
to use a derived class, optionally with additional code and data, to be used in
the measuring core. This is how the actual algorithm is implemented. We'll look
at this later.

On line 10 starts a field definition. Fields and getters/setters are generated
for the field.

Next comes the class controlling the whole measurement. 
\begin{lstlisting}[language=JAVA]
package ch.ethz.ruediste.roofline.measurementDriver.measurementControllers;

public class TriadMeasurementController implements IMeasurementController {

	public String getName() {
		return "triad";
	}

	public String getDescription() {
		return "runs the triad kernel";
	}

	@Inject
	public QuantityMeasuringService quantityMeasuringService;

	@Inject
	public RooflineController rooflineController;
	
	public void measure(String outputName) throws IOException {
	...
	}
}
\end{lstlisting}
The class implements IMeasurementController and has to be placed in the
measurementControllers package. The measure command will instantiate the class
and call the measure() method. The getName() method returns the name of the
measurement, which is used to identify the measurement.

The two fields with the @Inject attribute are initialzed by the dependency
injection framework when the class is instantiated. The quantity measuring
service allows to measure quantities like operation count, transferred bytes,
performance etc. The roofline controller manages a roofline plot. We will see
how these facilities are used when we look at the body of the measure function:

\begin{lstlisting}[language=JAVA]
public void measure(String outputName){
	rooflineController.setTitle("Triad");
	rooflineController.addDefaultPeaks();

	for (long size = 10000; size < 100000; size += 10000) {
		// initialize kernel
		TriadKernel kernel = new TriadKernel();
		kernel.setBufferSize(size);
		kernel.setOptimization("-O3");

		// add a roofline point
		rooflineController.addRooflinePoint(
			"Triad", Long.toString(size),
			kernel, Operation.CompInstr,
			MemoryTransferBorder.LlcRam);

		// measure the throughput
		Throughput throughput = quantityMeasuringService.measureThroughput(
			kernel, MemoryTransferBorder.LlcRam, ClockType.CoreCycles);

		// measure the operation count
		OperationCount operations = quantityMeasuringService
			.measureOperationCount(kernel, Operation.CompInstr);

		// print throughput and operation count
		System.out.printf("size %d: throughput: %s operations: %s\n", size,
			throughput, operations);
	}

	rooflineController.plot();
}
\end{lstlisting}

First the roofline plot is initialized with the title and the default peaks.
Then, for each buffer size, the kernel is initialized. For each kernel, the
optimization flags used to compile the kernel have to be specified.

Then the roofline controller is instructed to add a roofline point to the plot.
The first argument is the series name, next the label of the data point. Points
with the same series name are connected with a line in the plot. The rest of the
arguments specify the kernel and how the required quantities should be measured.

In the rest of the loop, the throughput and the operation count are measured and
printed to the console. This is an example of how to use the quantity measuring
service.

The last statement of the measure() body causes the plot to be output to a file
in the current directory. This involves the invocation of gnuplot.

During the invocation of addRooflinePoint() and the quantity measuring service a
lot was going on under the hood. First a measurement was created from the
kernel and the measurers required to measure the requested quantities. Then was
checked if there is already a result for the measurement in the cache. If not,
the measurement was serialized, the measuring core was configured, built
and started. Then the result of the core was parsed and stored in the cache.
And finally, the requested quantities were caculated.

The only measurement specific part involved in this process is the
implementation of the kernel. First the header:

\begin{lstlisting}[language=C++]
class TriadKernel : public TriadKernelData{
	double *a,*b,*c;
	
protected:
	std::vector<std::pair<void*,long> > getBuffers();

public:
	void initialize();
	void run();
	void dispose();
};
\end{lstlisting}

The kernel requires three buffers. All declared methods override methods
from the KernelBase. The buffers are allocated and initialized in
initialize() and freed in dispose(). getBuffers() returns the buffers along with
their sizes. This is used to clear the or warm the caches. run() contains the
actual algorithm. 


\begin{lstlisting}[language=C++]
void TriadKernel::initialize() {
	srand48(0);
	size_t size = getBufferSize() * sizeof(double);

	// allocate the buffers
	a = (double*) malloc(size);
	b = (double*) malloc(size);
	c = (double*) malloc(size);

	// initialize the buffers
	for (long i=0; i<getBufferSize(); i++){
		a[i]=drand48();
		b[i]=drand48();
		c[i]=drand48();
	}
}

std::vector<std::pair<void*, long> > TriadKernel::getBuffers() {
	size_t size = getBufferSize() * sizeof(double);

	std::vector<std::pair<void*, long> > result;
	result.push_back(std::make_pair((void*) a, size));
	result.push_back(std::make_pair((void*) b, size));
	result.push_back(std::make_pair((void*) c, size));
	return result;
}

void TriadKernel::run() {
	for (long p = 0; p < 1; p++) {
		for (long i = 0; i < getBufferSize(); i++) {
			a[i] = b[i] + 2.34 * c[i];
		}
	}
}

void TriadKernel::dispose() {
	free(a);
	free(b);
	free(c);
}

\end{lstlisting}

\subsection{Multi Language Infrastructure}
\label{sec:MultiLanguageInfrastructure}
The shared entities are used from both C++ and Java. To avoid having to
manually synchronize two versions of the same class, the source code for the C++
and the Java implementation is generated from an XML definition by the Shared
Entity Generator. The XML definition contains class and field definitions only,
no code. If class specific code is needed, it has to be implemented separately for each language and merged with the field definitions using
inheritance.

The shared entity definitions, written in XML, are parsed using a
serialization library called XStream. XStream maps classes to an XML
representation. The classes used to define the shared entities are shown
in \umlRef{MultiLanguageClassDefinition}.

\umlFloat{MultiLanguageClassDefinition}{Classes representing a multi language
class definition}

The following is an example of a class definition:
\begin{verbatim}
<?xml version="1.0" encoding="UTF-8"?>
<class name="MultiLanguageTestClass" 
  cBaseType="MultiLanguageObjectBase"
  javaBaseType=""
  comment="Multi Language Class used for unit tests">

  <field 
    name="longField" 
    type="long" 
    comment="test field with type 'long'"/>
  <list  
    name="referenceList" 
    type="MultiLanguageTestClass" 
    comment="list referencing full classes"/>
  <field 
    name="referenceField" 
    type="MultiLanguageTestClass" 
    comment="field referencing another class"/>
</class>
\end{verbatim}

After the definitions are loaded, Velocity templates are used to generate all
source code.

A normal entity has a C and a Java base type. The C base type has to directly or
indirectly inherit from SharedEntityBase, which is a polymorphic class.
This allows to use the RTTI (RunTime Type Information). Java base types have no
such constraint (due to the implicit common base class Object). The base types
are just included in the generated source code, but have no other effect on the
code generation.

A derived entity names another entity as base type. The C and
Java base types are set to that class. The fields of the base class are included in
the serialization process. If just the C and Java base types would be set to the
shared entity used as base class, the generated class would still derive
from the base class, but the fields of the base class would not be included in
the serialization process.

Often it is necessary to mix hand written code with the generated code. To
support this, a suffix can be specified, which is added to the name of the
generated class. Only the name of the generated class is affected, not the type
name used for references to the class. A class named without the specified
suffix has to be provided manually, and should derive from the generated class.
Any additional code as well as additional fields can be included in the hand
written class.

The class definitions and the generated code is located in the Multi Language
Classes project. The generated Java code is linked by the Measurement Driver
project. The generated C code is linked by the Measuring Core. The following
Diagram shows these dependencies:

\umlDiagram{MultiLanguageCodeGeneration}

\subsubsection{Serialization and Deserialization}
Along with the source code for each class, a service serializing and
deserializing multi language objects to/from a simple text based format is
generated for both languages. It supports the following primitive types:
\begin{itemize}
\item double
\item integer
\item long integer
\item boolean
\item string
\end{itemize}

References to other shared entities are supported. The serializer can handle
general object graphs.

Lists containing one of the supported primitive types as well as containing
references to other shared entities are supported.

The service implementations for both languages follow the same structure. Each
has two methods, one for serialization and one for deserialization. 

The serialization method receives an object and an output stream. The  method
body contains an if for each known serializable class, which checks if the class
of the object received is equal to the serializable class. If true, the value of
all fields of the class and it's base classes get serialized. For references,
the  serialization method is called recursively with the same output stream and
the  referenced object as parameters.

The deserialization method works analogous to the serialization method. It
receives an input stream. The method body contains an if for each known
serializable class, which checks if the next line of the input names  the
serializable class. If true, a new instance of the class is created and the
value of all fields are read from the input and set on the created instance,
including all fields declared in a base class. If a reference is encountered,
the deserialization method is called recursively with the same input, and the
returned instance is used as field value.

\subsection{Frontend}
The measurement tool is a console tool controlled using command line options.
Measurement results are either directly dispalyed on the console, dumped to a
data file or processed, usually for generating a graph. The graph is typically
stored as a file. But unlike normal tools, the source code is expected to change
frequently, and the user likely switches often between coding and using the
tool.

To support this usage pattern, the build process has been integrated into the
normal tool operation. The frontend is used to first trigger the build process
and then invoke the measurement driver. Otherwise, the user would have to keep
to console windowses open, one for building and one for measuring, and not to
forget building to see the changes made to the source code.

After building, the frontend starts the the measurement driver, forwarding it's
own command line. Certain flags are used to control the operation of the
fronted. These are not forwarded.

The frontend has a configuration system. The known configuration keys are
defined at the top of the Main class. There are three configuration sources. The
default configuration stored in a configuration file. It contains templates
which are expanded during the build process. The result is included as resource
in the generated JAR file. Flags of the default configuration can be overwritten
using a user configuration file, which is located by default under
"~/.roofline/frontendconfig". This location can be changed using a command line
argument. Finally, the command line options known by the frontend are used to
modify the configuration flags after they have been loaded. 

\subsection{Measurement Driver}
For the design of the measurement driver, we used software engineering best
practice, namely unit testing (junit), mocks (jmock), and dependency
injection (guice). Further a domain model (DOM), controllers, repositories and
stateless services as described in \cite{evans2004domain}.
Describing all these concepts lies beyond the scope of this report. It is
assumed that the reader has a basic understanding of the mentioned concepts.

The following diagram shows an overview of the driver:

\umlDiagram{measurementDriver/MeasurementDriverOverview}

In the following paragraphs, we will have a quick look at the look at the
different parts. 

The entry point of the driver is the \class{Main} class. First, the dependency
injection framework is initialized. This is accomplished using the
\class{MainModule}. It's \method{configure()} method uses the
\class{ClassFinder} to find all compiled classes and configures how they are
instantiated, mainly based on naming conventions.

Then the command line arguments are parsed. If command line auto completion is
desired (indicated by the '-autocomplete' flag), the auto completion process
starts. 

Otherwise the configuration is initialized, using the flags specified at the
command line and the configuration files (default configuration stored in the
jar and user configuration from the home directory of the user).

The last step in the initialization sequence is to set up log4j, the logging
framework. 

Then the class for the command given on the command line is instantiated and
the \method{execute()} method on the resulting \class{ICommand} is called.

When a 'measure' command is given, a \class{MeasurementCommandController} is
instantiated. The command controller looks a the next command line argument,
instantiates the corresponding measurement controller and calls the
\method{measure()} method.

The measurement controller will typically create multiple \class{Measurement}s
with different parameters. The \class{ParameterSpace} facilitates iterating
over all possible parameter combinations. Each parameter is associated to an
\class{Axis}. For each axis, one or multiple values can be given. The
\method{getAllPoints()} returns a \class{Coordinate} for each possible value
combination. Itearating over the points in the space, the measurement controller
can construct a measurement for each point. 

The results of the constructed measurements can be either printed to the
console, or stored in one of various \class{Plot}s. When all data is gathered,
the plot can be rendered and written to an output file using the
\class{PlotService}.

Although it is possible to directly create the \class{Measurer}s required to
measure something, most of the time the intent is to measure a certain
\class{Quantity} (\class{OperationCount}, \class{TransferredBytes},
\class{Performance} etc). The \class{QuantityMeasuringService} can be used to
obtain a \class{QuantityCalculator} for a quantity. The calculator can be
queried for the list of measurers which are required to calculate the quantity.
When the results of all required measurers are known, they can be passed to the
calculator, which will return the desired quantity. In addition, the quantity
measuring service provides convenience methods for working with the quantity
calculators.

Once the \class{Measurement} is constructed, the \method{measure()} method of
the  \class{MeasurementAppController} is used to perform the measurement. First
it is checked if a cached result is available for the measurement (using the
\class{CacheService}). If not, the \class{MeasuringCoreService} is used to build
the core for the measurement and to start the core.

The \class{Kernel}s can define macros. During build preparation, all macro
definitions present in the measurement are collected and written to generated
header files within the core.

During the execution of the measurement driver, the run time used for the
various tasks is collected using the \class{RuntimeMonitor}. At the end of the
execution, the times spent for the tasks is printed to the console.


\subsubsection{Dependency Injection Configuration}
Generally, a convention over configuration approach was chosen for the
configuration of the dependency injection. The conventions as well as optional
exceptions are defined in the MainModule. 
The conventions are:
\begin{description}
\item[Services] all classes in the services packages are bound to themselves as
singletons
\item[Repositories] all classes in the repositories packages are bound to
themselves as singletons
\item[Application Controllers] all classes in the 'appControllers' package are
bound to themselves as singletons
\item[Measurement Series] all classes deriving form IMeasurementSeries in the
measurement series package are bound to the IMeasurementSeries interface
annotated with their name
\item[Commands] all classes deriving from ICommand in the commands package are
bound to the ICommand interface annotated with their name
\item[Measurement Controllers] all classes deriving from IMeasurementController
in the measurement controller package are bound to the IMeasurementController
interface annotated with their name
\end{description}

\subsubsection{Configuration}
The design goal was to create a configuration system which
\begin{itemize}
\item allows to set configuration flags from the command line and from
configuration files
\item can manage some form of comment for the flags
\item makes the available flags transparent
\item supports user specific configuration
\end{itemize}

\umlDiagram{measurementDriver/Configuration}
The central class of our solution is the \class{ConfigurationKey}. A
configuration key contains a string key which identifies the configuration flag it represents. In
addition, it contains a description and the default value of the flag. It has a
template parameter which defines the data type of the flag. This removes the
necessity to use type casts when reading configuration flags. Configuration keys
should be stored in public static variables. The help command scans all classes
of the measurement driver for such configuration keys and prints the key string
and the description. After the configuration is loaded, it is checked if a
configuration key is defined for each configuration flag specified. If a
configuration key for a flag is missing (or more likely, a configuration flag
has been misspelled in the configuration) an error is generated.

The values associated with configuration keys are stored in
\class{Configuration}s. Configurations can be chained together using the parent
links. If no value is found in a configuration or all of its ancestors, the
default value stored in the configuration key is used. 

The state of a configuration kan be saved on a stack using \method{push()} and
restored using \method{pop()}. All modifications to a configuration after a push
are undone by the pop. This can be used to temporarily change the
configuration.

The following paragraphs describe the sources of configuration flag definitions
in order of decreasing precedence. 

The command line is scanned for arguments starting with a dash. Such arguments
are expected to be in the form of "-$<$flag key$>$=$<$value$>$" and specifies
that the configuration with the specified flag key should have the specified
value. Configuration flag definitions on the command line have highest
precedence.

Next come two configuration files. They both have the same format: each line
consists of the flag key, followed by an equal sign and the flag value. 

The first file is the user configuration file. By default it is located under
\textasciitilde/.roofline/config, but this can be changed using the
"userConfigFile" configuration flag, in particular by overwriting the flag on
the command line.

The second file is the default configuration. It is located in the source code
of the measurement driver, and can be loaded from the classpath. It contains
some placeholders, which are expanded during the build process.

Finally, the flag definitions with lowest precedence are the default values
given in the configuration keys.

\subsubsection{Auto Completion}

\subsubsection{Commands}
A command is represented by a class deriving from ICommandController and should
be placed in the commands package. A command has a name and a description, which
should be the return value of the getName() respectively getDescription()
methods of the command. The measurement driver expects a command name as first
argument. The name is matched against the names of all available commands. If a
command matches, the execute() method of a new instance of the corresponding
class is called with the remaining command line arguments as parameter.

\subsubsection{Measurement Controllers}
The operation of the measurement driver is controlled by the measurement
controllers. They define which measurements to perform and how to process the
output. The measure command instantiates a measurement controller and calls the
measure() method.

\subsubsection{Parameter Space}
When implementing measurement controllers, one often has to iterate over all
possible combinations of some parameters. The \class{ParameterSpace} was
designed to support this. 

Every parameter is identified by an \class{Axis}. For
each axis, one or multiple values are specified. After the desired values are
specified, all possible parameter combinations can be generated, represented by
\class{Coordinate} objects. The points are generated implicitely when iterating
over the space.

Example:
\begin{lstlisting}[language=Java]
space.add(systemLoadAxis, SystemLoad.Idle);
space.add(systemLoadAxis, SystemLoad.DiskOther);
space.add(systemLoadAxis, SystemLoad.DiskAll);
space.add(systemLoadAxis, SystemLoad.AddOther);
space.add(systemLoadAxis, SystemLoad.AddAll);

space.add(clockTypeAxis, ClockType.CoreCycles);
space.add(clockTypeAxis, ClockType.ReferenceCycles);
space.add(clockTypeAxis, ClockType.uSecs);

for (Coordinate coordinate : space) {

	ClockType clockType 
		= coordinate.get(clockTypeAxis);
	SystemLoad systemLoad 
		= coordinate.get(systemLoadAxis);
	...
}			
\end{lstlisting}

To faciliate the initialization of measurements, the classes of the measurement
description have an \method{intialize()} method which takes a coordinate as
parameter. Depending on the kernel or measurer at hand, some fields are
set to the value of an axis given by the coordinate.

The most common axes are defined in the \class{Axes} class.

\subsubsection{Retrieving Outputs}
To process the results of a measurement, it is frequently necessary to
retrieve the output of a specific measurer. 

The straight forward approach would be to give each measurer an unique id, and
to store the id of the measurer with the measurer output. Measurers are newly
created with each invocation of the measurement driver, possibly leading to new
ids. But for caching, the ids of the measurers do not matter.

To overcome these problems, two ids are generated for each measurer. One
identifier uniquely identifying each instantiated measurer. And an id which is
unique within one measurement. When loading a result from cache, the unique
identifiers of the loaded result are set to the identifiers of the measurement
at hand.

To retrieve the output of a measurer, the \class{MeasurementResult} and
the \class{MeasurementRunOutput} provide several methods which take a measurer
as argument and return it's output.

\umlDiagram{measurementDriver/RetrievingOutputs}

\subsubsection{Plotting}
Generating plots is an important feature of the tool. There are different plot
types. The most important are roofline plots and distribution plots. A
\class{RooflinePlot} contains all data required to generate a roofline plot,
including peak performances and memory bandwidths. 

A \class{DistributionPlot} is a 2D plot containing multiple data series. For
every discrete $x$ coordinate, there can be multiple values per series. The
values are shown using an error box for the 25th/75th percentile and whisker
bars for the minimum and maximum.

Once an instance of a \class{Plot} is filled with data, it can be passed to one
of the \method{plot()} methods of the \class{PlotService}. A gnuplot
script and the required data file will be created in the current
directory. Then gnuplot is called to generate a pdf of the plot.

\subsubsection{The MeasurementAppController}
The measurement application controller is the entry point for performing
measurements. It is the sole client to the \class{MeasuringCoreService}, which
provides the low level control over the measuring core, and keeps track of the
measurement the core is compiled for. 

In addition, it uses the \class{MeasurementHashRepository} to keep track of
measurements which have equal measuring cores.

The main method of the controller is measure(). Functionality in pseudo Code:

\begin{minipage}{\textwidth-18pt}
\begin{lstlisting}[language=Java,style=pseudoCode]
MeasurementResult measure(measurement, numberOfRuns)
"prepare measurement"
runOutputs=[]
if (useCachedResult || "measurement has been seen")
	loaded="load stored results"
	if (loaded!=null)
		if (!shouldCheckCoreHash
			|| currentCoreHash==loaded.coreHash)
			runOutputs=loaded
			
if ("more results needed")
	newResult=performMeasurement()
	"merge and store loaded and new run outputs"
	
"build and return MeasurementResult with the desired number of run outputs"
\end{lstlisting}
\end{minipage}

The method first tries to load a measurement result from the result cache. If
not enough run outputs are loaded (or none at all), the measurement is performed
to get the remaining measurement run outputs.

Finally, a measurement result with exactly the requested number of runs is
constructed and returned.

It is possible to disable loading stored results by setting the useCachedResults
configuration key to false. The measurement is performed and existing results
are overwritten.

The hash code of the core is stored along with the measurement result. This
allows to check if the currently compiled core is equal to the core which was
used during the measurement. By default, if the core changed since the results
were generated, the results are not used and new results are generated using the
current measuring core. By setting shouldCheckCoreHash to false, this check can
be skipped.

Preparing and building the measuring core are expensive operations in term of
runtime. Therefore, the measurement application controller keeps track of as
much information about the measurements and the cores needed to perform them as
possible. The \class{MeasurementHashRepository} is used for this purpose. It
has the following internal Model: 
\umlDiagram{measurementDriver/MeasurementHashRepositoryModel}

The \class{Core} class is private and does not leave the repository.

The model is exposed through the following methods
\begin{itemize}
  \item areCoresEqual(measurementA, measurementB): bool
  \item setHaveEqualCores(measurementA, measurementB)
  \item setCoreHash(measurement,coreHash)
  \item getCoreHash(measurement): CoreHash
\end{itemize}

Before building, the controller asks the repository if the core for the new
measurement is the same as the currently built one. (using
\method{areCoresEqual()}) If the cores are the same, no building is required.

During the build preparation the controller and the \class{MeasuringCoreService}
monitor changes to the core. If no changes were necessary, the repository is
notified using \method{setHaveEqualCores()}. The cores of the two measurements
are merged.

When a core hash is required (for example to check if the current core is the
same as the one used to generate a stored result), the controller first asks the
repository for it using \method{getCoreHash()}. If the hash is not known, the
core is built for the measurement, the hash is calculated from the core and
stored in the repository using \method{setCoreHash()}. This could again lead
to a merging of two cores, if a core with the same hash is present already.

Building the measuring core can become necessary when the core hash of a
measurement has to be known, or when a measurement is to be performed. This is
reflected in the call graph of the methods within the application controller:

\umlDiagram{measurementDriver/MACCallGraph}

Since it makes sense that services can start measurements, the
\class{MeasurementService} provides a \method{measure()} method. The service knows an
instance of \class{IMeasurementFacility} which provides \method{measure()}, and
forwards all calls to its own \method{measure()} method to the measurement
facility. The facility is is implemented by the \class{MeasurementAppController}
controller. Therefore the service ultimately forwards all calls to
\method{measure()} to the application controller.

\umlDiagram{measurementDriver/measureCallLift}


\subsubsection{Architecture Specific Behavior}
The measurement driver supports multiple system architectures. Currently, the
system architecture is identified by the available PMUs. When a performance
event is to be read, a list with the event for each architecture is passed to
the \method{getAvailableEvent()} method of the \class{SystemInfoService}. The
method returns the available event. In other cases, the presence of a PMU is
checked directly.

For further development, it might become beneficial to use the specification
pattern. 

\subsubsection{Preprocessor Macros}
Preprocessor macros are used to allow flexible compile time parameterization of
the measuring core. The macros are defined by the measurement driver. Before the
compilation of the measuring core, the measurement driver writes the definition
of each macro to a separate include file. This allows the build system to track
macro definition changes for and to recompile only the required parts.

In the measurement driver, each macro is identified by a macro key, which
contains the macro name, a description and the default value. The macro
definitions are stored in classes deriving from MacroDefinitionContainer. The
classes should define macro keys by placing them in private static variables. To
access the macro definition, getters and setters have to be provided. 

When the measuring core is configured to perform a measurement, the macro keys
are collected from the classes of the measurement driver using reflection. Then
the macro definitions are extracted from the measurement definition and
referenced objects. If no definition is given for a macro, the default
definition found in the macro key is used. If contradicting definitions are
found, an error is raised.

\umlDiagram{measurementDriver/MacroDefinitions}

\subsubsection{Measurement Result Caching}
The measurement controllers mix the definition of the measurement parameters and
the processing of the output. Thus, if the output processing logic needs to be
modified, the measurements have to be performed again. This causes a delay,
which is avoided by caching the measurement results. 

All parameters of a measurement are contained within the measurement description
and the referenced objects. Therefore, if the measurement description is
identical to a measurement description of a previous measurement, the result of
the previous measurement can be reused. 

The cache mechanism works using a hash function on the XML representation of the
measurement description. After a measurement has been performed, a file named
after the hash value of the measurement description of the measurement is created
and the measurement results are stored therein. Before a measurement is
performed, the hash value is computed. If a corresponding file is found, the
previous measurement results are reused.

\subsection{Measuring Core}
The measuring core is based on the object graph constructed by the driver. The
classes are extended with code and data.

\subsubsection{Core Architecture}
To fully utilize multi core systems, applications have to be implemented with
multiple threads or processes. Measuring such applications is considerably more
difficult than measuring single threaded applications. If the thread management
is implemented specifically for the measurement at hand, the measuring code can
be weaved into it by hand. But if the threads or processes are created within
legacy or closed source code, the measuring tool has to take care of detecting
the creation of threads and install the necessary measurers. For the roofline
measuring tool we will only consider multi threaded applications.

To gain full control over the kernel code, the measurement tool starts the
kernel within a child process. The parent process attaches to the child using
ptrace. This causes the child to be stopped when certain events occur and the
parent is notified. The events include thread creation and breakpoints.

Every kernel thread can raise events at any point. The events include thread
creation, breakpoints, starting and stopping of workloads etc. Whenever an event
occurs, the rule list has to be searched and the matching actions have to be
executed. The rule list is always searched in the thread which raised the event.
If the event caused the thread to be stopped and the parent thread to be
notified, the parent restarts the thread with a notification of the event which
occurred. The thread will search the rule list and continue execution.

It is important to distinguish the events from notifications in this context. An
event is handled by the rule list of the child and typically generated by the
child process as well. A notification is used to communicate between the child
process and the parent process. However, the parent can notify (using a
notification) the child of a certain observation, which causes the child to
generate an event.

The rule list is always searched in the thread generating the event. If an
action has to be executed in another thread, it has to be queued using
\method{ChildThread::queueAction()}

The technically most challenging problem is to interrupt another thread and make
it execute some action. There are two approaches to this problem, either using a
signal handler of the thread or using ptrace to call code within the thread.
Since installing a signal handler in a thread could cause unwanted side effects,
we use the ptrace approach. SIGTRAP is sent to the target thread, which will
cause it to be stopped. The parent modifies the thread state of the stopped
thread. When the thread resumes execution, it executes some event handling code
and then return to the location the thread was interrupted.

\subsubsection{Child Thread States}
\umlDiagram{measuringCore/ChildState}
The parent process manages a state for each child thread. When a child thread
starts, ptrace will immediately stop it with SIGSTOP. If the notification system
is ready, a ThreadStarted notification will be sent to the child.

While a notification is beeing processed by the child, it's state is set to
processing. When no more notifications are pending, the state is set to running.
If a notification is queued from one thread to another and the state of the
receiving thread is running, a SIGTRAP is sent to the receiving thread and the
state of the receiving thread is set to stopping. The stopping state indicates
that the thread will stop eventually.

A thread can exit at any time. When this is detected by the parent process, the
state and the notification queue are erased.

\subsection{Thread Representation in the Child Process}
In the child process, threads are represented as \class{ChildThread} objects.
Since every \class{Workload} runs in it's own thread, a child thread is
associated with every workload. 

When a new thread is spawned, it will be stopped by ptrace. The parent sends the
ThreadStarted notification to the child. In the handler, the child will
instantiate the \class{ChildThread}. If the started thread is a workload thread,
the startup routine of the workload will associate the instantiated child thread
with the workload.

Actions can be sent from one thread to another using
\method{ChildThread::queueAction()}. 

\subsubsection{Building}
Each measurement can be performed with different compiler optimization flags and
macro definitions. Therefore, the measuring core has to be rebuilt for each
measurement, which makes rebuilding the measuring core a frequent operation. It
should therefore be as fast as possible. This is achieved by carefully tracking
all build dependencies and by using ccache. 

CCache is a compiler cache. Whenever the compiler is run, ccache hashes all
input files, together with the compiler flags. It then checks if it's cache
already contains an entry for the hash value. If this is not the case, ccache
runs the compiler and stores the output together with the hash value of the
input in it's cache. If the hash value is present already, it does not run the
compiler but uses the compiler output stored in it's cache. This considerably
speeds up recompilations.

But ccache still has to build the hash values and copy the compiler output,
which takes some time. This is where tracking the build dependencies comes in.
The following parameters can change between measurements:
\begin{description}
\item[Macro definitions]
Each macro definition is stored in a separate file, which is only updated by the
measurement driver if the macro definition changes. Every source file which
needs a macro definition includes the corresponding file. These inclusions are
tracked and allow to only recompile the affected source files.
\item[Compiler flags] 
Each kernel specifies it's own optimization flags. The compiler flags used for
the rest of the measuring core do not affect the measurement results. The compiler
flags are stored in a separate file. Whenever it is changed, the parts affected
are recompiled.
\item[Compiled Kernels]
For each measurement, only the kernels actually used are compiled. The
measurement driver writes the kernel names into a separate file. The child
binary is recompiled when it changes.
\end{description}

The build process is controlled using the gnu make utility \cite{make}. Make
automatically determines which parts of a program have to be recompiled, based
on rules stored in a makefile. Each rule consists of target files, prerequisite
files and a recipe. Make checks the modification times of the target and the
prerequisite files. If any prerequisite is newer than any target, the recipe is
executed in order to update the target files. The recipe is a sequence of shell
commands.

The following diagram shows how the source files are categorized and compiled:

\umlDiagram{MeasuringCoreBuild}

The makefile used for the measuring core first instructs make to use the find
utility to get a list of all source files (with .cpp extension) in the
measuringCore/src and measuringCore/generated directories (ALL\_SOURCES).

Using the filter functions of make, the kernel sources are set to the subset of
ALL\_SOURCES which is located in src/kernels or generated/kernels. These are all sources related to kernels.
(ALL\_KERNEL\_SOURCES).

The sources of the parent process are the subset of ALL\_SOURCES which is
located in src/parent(PARENT\_SOURCES).

The source files of ALL\_SOURCES not contained in ALL\_KERNEL\_SOURCES or
PARENT\_SOURCES are stored in CHILD\_SOURCES.

The names of all present kernels are stored by the measurement driver in
generated/kernelNames.mk. For each of the kernels named there, the source file
named after the kernel and all source files in the subdirectory named after the
kernel collected in a variable (KERNEL\_SOURCES\_\$(kernel)). They are compiled
with the optimization flags of the compiler, wich are stored under
generated/kernelOptimization.

The sources of all current sources are collected and form, together with the
CHILD\_SOURCES, the sources of the child process.

The parent process is compiled from the PARENT\_SOURCES.

There is a rule without recipe with the kernel objects as target and the file
containing the measurement specific optimization flags as prerequisite. This
causes the file containing the optimization flags to be added as prerequisite
for each kernel object file.

In the C programming language, it is possible to include other files in a source
file. Of course, the compiled code depends on the contents of the included
files, too. To track these build dependencies, the compiler is instructed to
generate rules without recipes with the object file as target and the source
file together with the included files as prerequisites. The generated rules are
stored in .d files in the build directory and are included in the makefile.

If special compilation flags are required for a source file, a rule should be
added near the end of the makefile.

\subsubsection{System Initialization}
We chose a modular approach to initialize the measuring core. Whenever a system
part needs to run code when the program starts or shuts down, it can instantiate
a class derived from SystemInitializer.

This is preferably achieved by declaring a global static variable named dummy
in a .cpp file. Example:
\begin{verbatim}
// define and register a system initializer.
static class FooInitializer: public SystemInitializer{
  void start(){
    // code to be executed on startup
  }

  void stop(){
    // code to be executed on shutdown
  }
} dummy;
\end{verbatim}

It is important to give every initializer subclass an individual name. Use the
name of the file the initializer is declared in as prefix. If two initializer
classes have the same name, they don't work correctly (instances of the wrong
classes are created)

Whenever a SystemInitializer is instantiated, the instance is registered in a
static global list. On system startup and shutdown, the start() respective
stop() method of all registered SystemInitializers is called.

\section{How To}
\subsection{Installation}
see INSTALL file in tool directory

\subsection{Create a New Kernel}
First, you have to create an XML description of the new kernel in
sharedEntities/definitions/kernels. Look at some of the other files in that
directory and choose one as starting point for your own. Copy the chosen file
and give it the name of your kernel. It must end with 'Kernel'. The file name is
taken as class name for your kernel description.

Next, run 'rot help' to generate the source code from your XML description. In
case your class defines a 'javaSuffix', the compilation following the source
generation will fail. Create the Java code of your kernel class in
measurementDriver/src/ch/ethz/ruediste/roofline/sharedEntities/Kernel and make
it inherit from the generated class with the suffix. (see other kernels for
examples)

The Java part of your kernel is now ready. You can use it in a measurement. But
we are still missing the implementation in the measuring core. To implement the
kernel, it is indispensable to specify a 'cSuffix' in your kernel description.

Create a class in measuringCore/src/sharedEntities/kernels, include the
generated header file (filename contains the suffix) and derive from the
generated class. Add fields for all the data buffers you plan to use
(if any). Override and implement the following methods:
\begin{description}
\item[\method{initialize()}] Allocate and initialize the required buffers.
\item[\method{getBuffers()}] Return the list of all buffers along with their
size. This is used to automatically clear or warm the caches.
\item[\method{run()}] Run the kernel.
\item[\method{dispose()}] Free all buffers
\end{description}

In case you have additional code, you can place it in a subdirectory with the
same name as your kernel, without the 'Kernel' suffix. All '.cpp' files will be
compiled and linked with the measuring core.

\subsection{Use the Driver as a Library}
You can use the measurement driver as a library. First, run './gradlew
--daemon measurementDriver:runnableJar' to create a jar containing the driver
together with all dependencies. It will be located in
'measurementDriver/build/distributions/measurementDriver.jar'. Include the jar
in your project.

During the startup of your application, call LibraryMain.initialize(). Now you
can use the driver. You can directly instantiate entities and use
\method{Instantiator.instance.getInstance()} to retrieve service instances.

\subsection{Create New Measurement}
\begin{itemize}
\item create new class in measurementDriver/measurements
\item implement IMeasurement
\end{itemize}

\subsection{Add Configuration Key}
The configuration is used to set various flags in the measurement driver.
\begin{itemize}
\item add public static field of type ConfigurationKey to any class within the
measurement driver.
\end{itemize}

\subsection{Generate Annotated Assembly}
\begin{itemize}
\item in Eclipse, hit build (Ctrl+B)
\item change the kernel header file (make it recompile)
\item build again
\item from the console window, copy the compiler invocation for
MeasurementSchemeRegistration.cpp
\item open a terminal and go to tool/measuringCore/Debug.
\item paste the compiler invocation
\item insert "-Wa,-ahl=ass.s", check optimization flags
\item issue command
\item the annotated assembly code can be found in tool/measuringCore/Debug/ass.s
\item open the annotated assembly code in Eclipse
\end{itemize}

\section{Results on the Intel Core}
\label{sec:resultsIntelCore}
We repeated our measurements on a desktop computer with an Intel Core 2 CPU. The
results are not discussed in detail, but we tried to point out the most
important differences to the results we measured for the Yonah architecture
based system.

The CPU is an 'Intel Core 2 6600' (Family 6, Model 15, Stepping 6) running at
2.4 GHz. It contains two cores, each having a 32KB instruction and a 32KB data
L1 cache, 8 ways set associative, with 64 bytes line size. The two cores share a
4MB unified L2 cache, 16 ways set associative and with 64 bytes line size. The
TLB has 256 entries, 4 way set associative. The core frequency can scale between
1.6GHz and 2.4GHz. The bus frequency is 266MHz.

The main memory consists of two 1GB and two 512MB DDR2 modules, totaling in 3GB
available memory. The theoretical throughput of the memory is 8.3 GB/s, which is
3.46 Bytes per core cycle, if the CPU runs at 2.4GHz.

\subsection{Performance Counters}
We used 'coreduo::UNHALTED\_CORE\_CYCLES' for measuring time. For the operation
count, we used the follwing definitions for operation: 
\begin{description}
\item[SinglePrecisionFlop] SSE single precision operations.
\\{\footnotesize
'core::SSE\_COMP\_INSTRUCTIONS\_RETIRED:SCALAR\_SINGLE'\\
+2*'core::SSE\_COMP\_INSTRUCTIONS\_RETIRED:PACKED\_SINGLE'}
\item[DoublePrecisionFlop] SSE double precision operations.
\\{\footnotesize
'core::SSE\_COMP\_INSTRUCTIONS\_RETIRED:SCALAR\_DOUBLE'\\
+2*'core::SSE\_COMP\_INSTRUCTIONS\_RETIRED:PACKED\_DOUBLE'}
\item[CompInstr] Computational instructions retired. Counts SSE instructions
and x87 instructions. Used for x87 code. 
\\{\footnotesize 'coreduo::FP\_COMP\_OPSEXE'}
\item[SSEFlop] SSE operations, sum of SinglePrecisionFlop and
DoublePrecisionFlop
\end{description}

We used two variants of measuring the memory transfer volume. The {\bf MemBus}
variant uses 64*'core::BUS\_TRANS\_MEM', which measures the transfers on the
system bus. The {\bf MemL2} variant uses the counters for the L2 cache line
allocation and eviction, namely
64*('core::L2\_LINES\_IN:SELF'+'core::L2\_M\_LINES\_OUT:SELF'), combined
with 8*'core::SSE\_PRE\_EXEC:STORES' to take non temporal stores
into account.

We used XUbuntu 11.10, running a Linux 3.0.0-16 kernel in 64 bit mode and GCC
4.6.1.

\subsection{Performance and Bandwidth Ceilings}
For the Core architecture, one double precision addition and multiplication can
be issued every cycle. The latencies are 3 cycles for
the addition and 5 cycles for the multiplication. \cite{inteloptimize} 

The throughput and latencies of the packed SSE instructions are equal to those
of the x87 instructions. Thus with the x87 instruction set the theoretical
ceiling is 1 flop/cycle for pure additions and multiplications. With the SSE
instruction set the ceiling is 2 flop/cycle.

It should be possible to use the multiplier and the adder at the same time, so
if there is an equal number of additions and multiplications it should be
possible to reach 4 flop/cycle.

We used the same micro benchmarks for addition and multiplication as for the
Yonah architecture. 

For the mixed instruction set a different workload is used. Due to the latencies
we always have to keep 5 multiplications and 3 additions in flight. This is
achieved by the following code:

\begin{lstlisting}
#define MULTIPLICATIONS 5
#define ADDITIONS 3
double mulR[MULTIPLICATIONS];
double addR[ADDITIONS];
...
#define MUL for (int h = 0; h < MULTIPLICATIONS; h++)\
				mulR[h] *= base;
#define ADD for (int h = 0; h < ADDITIONS; h++)\
				addR[h] += 1;
for (long i = 0; i < iterations; i++) {
	for (int p = 0; p < UNROLL; p++) {
		MUL ADD ADD MUL ADD ADD MUL ADD
	}
}
\end{lstlisting}


Optimizing the value for UNROLL and DLP in the range of 1 to 19 we found the
following results:

\begin{center}
\begin{tabular}{llllr@{.}lr@{.}l}
\toprule
Operation&Instruction Set & DLP & UNROLL &  \multicolumn{4}{l}{Performance
[Flop/Cycle]} \\ \cmidrule{5-8}
&&&&\multicolumn{2}{l}{Measured}&\multicolumn{2}{l}{Theoretical}\\
\midrule
ADD   & x87       &  9 & 15 & 0&998 & 1&0\\
      & SSEScalar &  9 & 15 & 0&998 & 1&0\\
      & SSE       &  9 & 17 & 1&996 & 2&0\\
MUL   & x87       & 10 & 15 & 1&998 & 1&0\\
      & SSEScalar & 10 & 15 & 0&998 & 1&0\\
      & SSE       &  9 & 15 & 1&996 & 2&0\\
Mixed & x87       &  - & 10 & 1&996 & 2&0\\
      &SSEScalar  &  - &  9 & 1&995 & 2&0\\
      &SSE        &  - & 11 & 3&991 & 4&0\\
\bottomrule
\end{tabular}
\end{center}

In all cases we almost reach the theoretical values.

In our plots we show 2 flop per cycle as balanced scalar ceiling, 4 flops
per cycle as balanced SSE ceiling and 8 flops per cycle as dual core SSE
ceiling.

The bandwidth ceiling can be calculated from the parameters of the memory
subsystem and is 3.46 bytes per cycle (see section
\ref{sec:resultsIntelCore}).

We use the same microbenchmarks as for the Yonah architecture. Optimizing loop
unrolling and stream count resulted in  the following numbers:

\begin{center}
\begin{tabular}{lllr@{.}l}
\toprule
Operation & Streams & UNROLL &  \multicolumn{2}{l}{Throughput
[Byte/Cycle]} \\
\midrule
Load  &  1 & 1 & 2&58\\
Write &  1 & 3 & 1&63\\
Non Temporal & 1 & 4 & 2&02\\
\bottomrule
\end{tabular}
\end{center}

The results reported here were obtained using MemBus, but the results of the
MemL2 variant are almost identical.
 
In the stream benchmark \cite{stream} the system reaches a copy rate of 3177
MB/s, which is $2*3434MB/1.8GHz=2.84$ bytes per cycle. Our microbenchmark
reported a smaller number, but the result is comparable to ours.

We also measure the random access throughput. A large buffer is allocated and
randomly accessed. The troughput is 0.455 bytes per cycle. Since a cache line is
64 bytes, this results in one memory access every $64/0.455=141$ cycles.

In our roofline plots we only show the load ceiling (2.2 flops per cycle) and
the random access ceiling (0.45 bytes per cycle).

\subsection{Measurement and Validation}
The accuracy and precision are very similar to those of the Yonah architecture
based system.

\graphFloatTwo{Core: Ratio of the Actual Transfer Volume to
the Expected Transfer Volume} {core/valTBValues}{MemBus}
{core/valTBALTValues}{MemL2}

\graphFloatTwo{Core: Error of the Transfer Volume: $\err(\frac{\tau}{\Theta})$}
{core/valTBError}{MemBus}
{core/valTBALTError}{MemL2}

\graphFloat{core/valTBALTFlushValues}{Core: MemL2: Memory Transfer during Cache
Flush after kernel execution}

\graphFloatTwo{Core: Transferred Bytes of the Arithmetic Kernels}
{core/valTBArithTBValues}{MemBus}
{core/valTBALTArithTBValues}{MemL2}

\graphFloat{core/valTimeArithValues}{Core: Execution Time of the Arithmetic
Kernels} 
\graphFloat{core/valTimeArithError}{Core: Error of the Execution Time of the
Arithmetic Kernels: $\err\left(\frac{\text{Execution Time}}{\min(\text{Execution
Time})}\right)$}

\graphFloat{core/valTimeMemValues}{Core: Execution Time of the memory kernels} 
\graphFloat{core/valTimeMemError}{Core: Error of the Execution Time of the
Memory Intensive Kernels:  $\err\left(\frac{\text{Execution Time}}{\min(\text{Execution
Time})}\right)$} 

\graphFloat{core/valOpValues}{Core: Operation Count: $\frac{\text{Actual
Operation Count}}{\text{Expected Operation Count}}$}

\graphFloatTwo{Core: Error of the Transfer Volume with $K=10$:
$\err\left(\frac{\tau_{10}}{\Theta}\right)$} {core/valTBMinError}{MemBus}
{core/valTBALTMinError}{MemL2} 

\graphFloat{core/valTimeMemMinError}{Core: Error of the Execution Time with
$K=10$: $\err\left(\frac{\text{Execution
Time}_{10}}{\min(\text{Execution Time})}\right)$}

\graphFloat{core/valTimeArithMinError}{Core: Error of Execution Time with
$K=10$: $\err\left(\frac{\text{Execution
Time}_{10}}{\min(\text{Execution Time})}\right)$}

\graphFloat{core/valOpMinError}{Core: Error of Operation Count with
$K=10$: $\err(\frac{\text{Actual Operation Count}_{10}}{\text{Expected
Operation Count}})$}

\clearpage 
\graphFloatTwo{Core: Ratio of the Actual Transfer Volume to the Expected
Transfer Volume for two Read Kernels running in Parallel (using one Buffer per Kernel)}
{core/valTBThReadValues}{MemBus} {core/valTBALTThReadValues}{MemL2} 

\graphFloatTwo{Core: Ratio of the Actual Transfer Volume to the Expected
Transfer Volume for two Write Kernels running in Parallel (using one Buffer per Kernel)}
{core/valTBThWriteValues}{MemBus}
{core/valTBALTThWriteValues}{MemL2} 

\graphFloatTwo{Core: Ratio of the Actual Transfer Volume to the Expected
Transfer Volume for two Write Kernels using Streaming Stores running in Parallel (using
one Buffer per Kernel)} 
{core/valTBThWriteStreamValues}{MemBus}
{core/valTBALTThWriteStreamValues}{MemL2}

\graphFloatTwo{Core: Ratio of the Actual Transfer Volume to the Expected
Transfer Volume for two Triad Kernels running in Parallel (using one Buffer per Kernel)}
{core/valTBThTriadValues}{MemBus}
{core/valTBALTThTriadValues}{MemL2} 

\graphFloat{core/valTBThReadPoint}{Core: Distribution of the Results of the Read
Kernel using MemBus} 

\graphFloat{core/valTBThWritePoint}{Core: Distribution of the Results of the
Write Kernel using MemBus}

\graphFloat{core/valTBThWriteStreamPoint}{Core: Distribution of the Results of
the Write Kernel using Streaming Stores measured with the MemBus measurement
variant}

\graphFloat{core/valTBThTriadPoint}{Core: Distribution of the Results of the
Triad Kernel using MemBus} 

\graphFloat{core/valTimeThAddArithValues}{Core: Execution Time of two ADD
Kernels running at the same time}

\graphFloat{core/valTimeThReadMemValues}{Core: Execution Time of two Read
Kernels running at the same time}

\graphFloat{core/valTimeThWriteMemValues}{Core: Execution Time of two Write
Kernels running at the same time}

\graphFloat{core/valTimeThTriadMemValues}{Core: Execution Time of two Triad
Kernels running at the same time}

\graphFloat{core/valOpThValues}{Core: Ratio of the Actual Operation Count to
the Expected Operation Count for two ADD Kernels running at the same time}

\clearpage

\graphFloatTwo{Core: Error of the Transfer Volume of two Read Kernels running at
the same time with $K=10$: $\err\left(\frac{\tau_{10}}{\Theta}\right)$}
{core/valTBThReadMinError}{MemBus} {core/valTBALTThReadMinError}{MemL2} 

\graphFloatTwo{Core: Error of the Transfer Volume of two Write Kernels running
at the same time with $K=10$: $\err\left(\frac{\tau_{10}}{\Theta}\right)$}
{core/valTBThWriteMinError}{MemBus} {core/valTBALTThWriteMinError}{MemL2} 

\graphFloatTwo{Core: Error of the Transfer Volume of two Write Kernels using
Streaming Stores running at the same time with $K=10$:
$\err\left(\frac{\tau_{10}}{\Theta}\right)$}
{core/valTBThWriteStreamMinError}{MemBus}
{core/valTBALTThWriteStreamMinError}{MemL2}

\graphFloatTwo{Core: Error of the Transfer Volume of two Triad Kernels running
at the same time with $K=10$: $\err\left(\frac{\tau_{10}}{\Theta}\right)$}
{core/valTBThTriadMinError}{MemBus} {core/valTBALTThTriadMinError}{MemL2} 

\graphFloat{core/valTimeThReadMemMinError}{Core: Error of the Execution Time of
two Read Kernels running at the same time with $K=10$: $\err\left(\frac{\text{Execution
Time}_{10}}{\min(\text{Execution Time})}\right)$}

\graphFloat{core/valTimeThWriteMemMinError}{Core: Error of the Execution Time of
two Write Kernels running at the same time with $K=10$: $\err\left(\frac{\text{Execution
Time}_{10}}{\min(\text{Execution Time})}\right)$}

\graphFloat{core/valTimeThWriteStreamMemMinError}{Core: Error of the Execution
Time of two Write Kernels running at the same time with $K=10$: $\err\left(\frac{\text{Execution
Time}_{10}}{\min(\text{Execution Time})}\right)$}

\graphFloat{core/valTimeThTriadMemMinError}{Core: Error of the Execution Time of
two Triad Kernels running at the same time with $K=10$:
$\err\left(\frac{\text{Execution Time}_{10}}{\min(\text{Execution Time})}\right)$}

\graphFloat{core/valTimeThAddArithMinError}{Core: Error of the Execution Time of
two ADD Kernels running at the same time with $K=10$: $\err\left(\frac{\text{Execution
Time}_{10}}{\min(\text{Execution Time})}\right)$}

\graphFloat{core/valOpThMinError}{Core: Error of the Operation Count of two ADD
Kernels running at the same time with $K=10$:
$\err(\frac{\text{Actual Operation Count}_{10}}{\text{Expected Operation Count}})$}

\clearpage
\subsection{Experimental Results}
\subsubsection{BLAS}
\graphFloat{core/daxpy}{Core: Roofline Plot of the Vector-Vector Multiplication
(MemL2, DoublePrecisionFlop)} 
\graphFloat{core/daxpyWarm}{Core: Roofline Plot of the influence of the
initial cache state on the Vector-Vector Multiplication (MemL2,
DoublePrecisionFlop)}
\graphFloat{core/dgemv}{Core: Roofline Plot of the Matrix-Vector Multiplication
(MemL2, DoublePrecisionFlop)} 
\graphFloat{core/dgemvWarm}{Core: Roofline Plot of the Matrix-Vector
Multiplication (MemL2, DoublePrecisionFlop)}
\graphFloat{core/mmm}{Core: Roofline Plot of the Matrix-Matrix Multiplication
(MemL2)}
\graphFloat{core/MMMwarm}{Core: Roofline Plot of the Matrix-Matrix
Multiplication with Different Initial Cache States (MemL2)}

\graphFloat{core/MMMwarmInt}{Core: Effect of the Initial Cache State on the
Operational Intensity for the Matrix-Matrix Multiplication (MemL2)}

\clearpage
\subsubsection{FFT}
\graphFloat{core/fft}{Core: Roofline Plot of four FFT implementations (MemL2)}
\graphFloat{core/FFTwarm}{Core: Influence of the initial cache state (MemL2)}


\subsubsection{WHT}
\graphFloat{core/wht}{Core: Roofline Plot WHT (MemL2)}
\graphFloat{core/whtInt}{Core: Operational Intensity WHT (MemL2)}

\clearpage
\newpage
\section{List of Figures}
\renewcommand{\listfigurename}{}
\vskip -1cm
\listoffigures

\newpage
\renewcommand\refname{\vskip -1cm}
\section{References}
\bibliographystyle{abbrv}
\bibliography{report}

\end{document}
